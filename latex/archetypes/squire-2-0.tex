% Pandoc LaTeX template for checklists with basic PDF forms
\documentclass[11pt]{article}
\usepackage[margin=1in]{geometry}
\usepackage{hyperref}
\usepackage{amssymb}
% Optional AcroTeX eForms for richer widgets
\usepackage{eforms}
\hypersetup{colorlinks=true,linkcolor=blue,urlcolor=blue}
\usepackage{enumitem}
\setlist[itemize]{left=1.2em}
\setlist{noitemsep}
\def\tightlist{}

% Abstraction layer for form widgets
  % eForms uses \checkBox and \textField
  \newcommand{\mkCheckBox}[2][]{\checkBox[#1]{#2}}
  \newcommand{\mkTextField}[2][]{\textField[#1]{#2}}

% Enable form environment for interactive fields
% Usage in content requires raw LaTeX or a pandoc Lua filter to insert \CheckBox/\TextField
\begin{document}

\begin{center}
{\LARGE }\\[4pt]
\normalsize 
\end{center}
\vspace{1em}

% Begin PDF form region
\begin{Form}

\section{SQUIRE 2.0 Checklist}\label{squire-2.0-checklist}

\begin{quote}
Scope: Standards for QUality Improvement Reporting Excellence.

Reference: See \texttt{source/archetypes/squire-2-0.yml} for canonical
link and provenance.
\end{quote}

\subsection{Instructions}\label{instructions}

\begin{itemize}
\tightlist
\item
  Use the boxes to confirm each reporting item.
\item
  Add reviewer notes under each section as needed.
\end{itemize}

\subsection{Title and Abstract}\label{title-and-abstract}

\begin{itemize}
\tightlist
\item[$\square$]
  \textbf{1. Title:} Indicate that the manuscript concerns an initiative
  to improve healthcare.
\item[$\square$]
  \textbf{2. Abstract:} Provide a structured summary of the project.
\end{itemize}

\subsection{Introduction}\label{introduction}

\begin{itemize}
\tightlist
\item[$\square$]
  \textbf{3. Problem Description:} Nature and significance of the local
  problem.
\item[$\square$]
  \textbf{4. Available Knowledge:} Summary of what is currently known
  about the problem, including relevant previous studies.
\item[$\square$]
  \textbf{5. Rationale:} Informal or formal frameworks, models,
  concepts, and/or theories used to explain the problem.
\item[$\square$]
  \textbf{6. Specific Aims:} Purpose of the project and of this report.
\end{itemize}

\subsection{Methods}\label{methods}

\begin{itemize}
\tightlist
\item[$\square$]
  \textbf{7. Context:} Contextual elements considered important at the
  outset of introducing the intervention(s).
\item[$\square$]
  \textbf{8. Intervention(s):} Description of the intervention(s) in
  sufficient detail that others could reproduce it.
\item[$\square$]
  \textbf{9. Study of the Intervention(s):} Approach chosen for
  assessing the impact of the intervention(s).
\item[$\square$]
  \textbf{10. Measures:} Measures chosen for studying processes and
  outcomes of the intervention(s).
\item[$\square$]
  \textbf{11. Analysis:} Qualitative and quantitative methods used to
  draw inferences from the data.
\item[$\square$]
  \textbf{12. Ethical Considerations:} Ethical aspects of implementing
  and studying the intervention(s).
\end{itemize}

\subsection{Results}\label{results}

\begin{itemize}
\tightlist
\item[$\square$]
  \textbf{13. Results:} Initial steps of the intervention(s) and their
  evolution over time.
\item[$\square$]
  \textbf{14. Summary:} Key findings, including relevance to the
  rationale and specific aims.
\end{itemize}

\subsection{Discussion}\label{discussion}

\begin{itemize}
\tightlist
\item[$\square$]
  \textbf{15. Interpretation:} Nature of the association between the
  intervention(s) and the outcomes.
\item[$\square$]
  \textbf{16. Limitations:} Limits to the generalizability of the work.
\item[$\square$]
  \textbf{17. Conclusions:} Usefulness of the work.
\end{itemize}

\subsection{Other Information}\label{other-information}

\begin{itemize}
\tightlist
\item[$\square$]
  \textbf{18. Funding:} Sources of funding that supported this work.
\end{itemize}

{Notes}

\subsection{Provenance}\label{provenance}

\begin{itemize}
\tightlist
\item
  Source: See sidecar metadata in
  \texttt{source/archetypes/squire-2-0.yml}
\item
  Version: 2.0
\item
  License: Custom; see website
\end{itemize}

\end{Form}

\end{document}
