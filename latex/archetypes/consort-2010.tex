% Pandoc LaTeX template for checklists with basic PDF forms
\documentclass[11pt]{article}
\usepackage[margin=1in]{geometry}
\usepackage{hyperref}
\usepackage{amssymb}
% Optional AcroTeX eForms for richer widgets
\usepackage{eforms}
\hypersetup{colorlinks=true,linkcolor=blue,urlcolor=blue}
\usepackage{enumitem}
\setlist[itemize]{left=1.2em}
\setlist{noitemsep}
\def\tightlist{}

% Abstraction layer for form widgets
  % eForms uses \checkBox and \textField
  \newcommand{\mkCheckBox}[2][]{\checkBox[#1]{#2}}
  \newcommand{\mkTextField}[2][]{\textField[#1]{#2}}

% Enable form environment for interactive fields
% Usage in content requires raw LaTeX or a pandoc Lua filter to insert \CheckBox/\TextField
\begin{document}

\begin{center}
{\LARGE }\\[4pt]
\normalsize 
\end{center}
\vspace{1em}

% Begin PDF form region
\begin{Form}

\section{CONSORT 2010 Checklist}\label{consort-2010-checklist}

\begin{quote}
Scope: Consolidated Standards of Reporting Trials.

Reference: See \texttt{source/archetypes/consort-2010.yml} for canonical
link and provenance.
\end{quote}

\subsection{Instructions}\label{instructions}

\begin{itemize}
\tightlist
\item
  Use the boxes to confirm each reporting item.
\item
  Add reviewer notes under each section as needed.
\end{itemize}

\subsection{Title and Abstract}\label{title-and-abstract}

\begin{itemize}
\tightlist
\item[$\square$]
  \textbf{1a. Title:} Identification of the trial as randomised.
\item[$\square$]
  \textbf{1b. Abstract:} Structured summary of trial design, methods,
  results, and conclusions.
\end{itemize}

\subsection{Introduction}\label{introduction}

\begin{itemize}
\tightlist
\item[$\square$]
  \textbf{2a. Background:} Scientific background and explanation of
  rationale.
\item[$\square$]
  \textbf{2b. Objectives:} Specific objectives or hypotheses.
\end{itemize}

\subsection{Methods}\label{methods}

\begin{itemize}
\tightlist
\item[$\square$]
  \textbf{3a. Trial design:} Description of trial design (e.g.,
  parallel, factorial) including allocation ratio.
\item[$\square$]
  \textbf{3b. Changes to trial design:} Important changes to methods
  after trial commencement (such as eligibility criteria), with reasons.
\item[$\square$]
  \textbf{4a. Participants:} Eligibility criteria for participants.
\item[$\square$]
  \textbf{4b. Study settings:} Settings and locations where the data
  were collected.
\item[$\square$]
  \textbf{5. Interventions:} The interventions for each group with
  sufficient details to allow replication, including how and when they
  were actually administered.
\item[$\square$]
  \textbf{6a. Outcomes:} Completely defined pre-specified primary and
  secondary outcome measures, including how and when they were assessed.
\item[$\square$]
  \textbf{6b. Changes to outcomes:} Any changes to trial outcomes after
  the trial commenced, with reasons.
\item[$\square$]
  \textbf{7a. Sample size:} How sample size was determined.
\item[$\square$]
  \textbf{7b. Interim analyses and stopping guidelines:} When
  applicable, explanation of any interim analyses and stopping
  guidelines.
\item[$\square$]
  \textbf{8a. Randomisation: sequence generation:} Method used to
  generate the random allocation sequence.
\item[$\square$]
  \textbf{8b. Randomisation: type:} Type of randomisation; details of
  any restriction (e.g., blocking and block size).
\item[$\square$]
  \textbf{9. Allocation concealment mechanism:} Mechanism used to
  implement the random allocation sequence (e.g., sequentially numbered
  containers), describing any steps taken to conceal the sequence until
  interventions were assigned.
\item[$\square$]
  \textbf{10. Implementation:} Who generated the random allocation
  sequence, who enrolled participants, and who assigned participants to
  interventions.
\item[$\square$]
  \textbf{11a. Blinding:} If done, who was blinded after assignment to
  interventions (e.g., participants, care providers, those assessing
  outcomes) and how.
\item[$\square$]
  \textbf{11b. Similarity of interventions:} If relevant, description of
  the similarity of interventions.
\item[$\square$]
  \textbf{12a. Statistical methods:} Statistical methods used to compare
  groups for primary and secondary outcomes.
\item[$\square$]
  \textbf{12b. Additional analyses:} Methods for additional analyses,
  such as subgroup analyses and adjusted analyses.
\end{itemize}

\subsection{Results}\label{results}

\begin{itemize}
\tightlist
\item[$\square$]
  \textbf{13a. Participant flow (a diagram is strongly recommended):}
  For each group, the numbers of participants who were randomly
  assigned, received intended treatment, and were analysed for the
  primary outcome.
\item[$\square$]
  \textbf{13b. Losses and exclusions:} For each group, losses and
  exclusions after randomisation, together with reasons.
\item[$\square$]
  \textbf{14a. Recruitment:} Dates defining the periods of recruitment
  and follow-up.
\item[$\square$]
  \textbf{14b. Reason for stopping:} Why the trial ended or was stopped.
\item[$\square$]
  \textbf{15. Baseline data:} A table showing baseline demographic and
  clinical characteristics for each group.
\item[$\square$]
  \textbf{16. Numbers analysed:} For each group, number of participants
  (denominator) included in each analysis and whether the analysis was
  by original assigned groups.
\item[$\square$]
  \textbf{17a. Outcomes and estimation:} For each primary and secondary
  outcome, results for each group, and the estimated effect size and its
  precision (such as 95\% confidence interval).
\item[$\square$]
  \textbf{17b. Binary outcomes:} For binary outcomes, presentation of
  both absolute and relative effect sizes is recommended.
\item[$\square$]
  \textbf{18. Ancillary analyses:} Results of any other analyses
  performed, including subgroup analyses and adjusted analyses,
  distinguishing pre-specified from exploratory.
\item[$\square$]
  \textbf{19. Harms:} All important harms or unintended effects in each
  group.
\end{itemize}

\subsection{Discussion}\label{discussion}

\begin{itemize}
\tightlist
\item[$\square$]
  \textbf{20. Limitations:} Trial limitations, addressing sources of
  potential bias, imprecision, and, if relevant, multiplicity of
  analyses.
\item[$\square$]
  \textbf{21. Generalisability:} Generalisability (external validity,
  applicability) of the trial findings.
\item[$\square$]
  \textbf{22. Interpretation:} Interpretation consistent with results,
  balancing benefits and harms, and considering other relevant evidence.
\end{itemize}

\subsection{Other Information}\label{other-information}

\begin{itemize}
\tightlist
\item[$\square$]
  \textbf{23. Registration:} Registration number and name of trial
  registry.
\item[$\square$]
  \textbf{24. Protocol:} Where the full trial protocol can be accessed,
  if available.
\item[$\square$]
  \textbf{25. Funding:} Sources of funding and other support (such as
  supply of drugs), role of funders.
\end{itemize}

\subsubsection{Notes}\label{notes}

{Reviewer notes}

\subsection{Provenance}\label{provenance}

\begin{itemize}
\tightlist
\item
  Source: See sidecar metadata in
  \texttt{source/archetypes/consort-2010.yml}
\item
  Version: 2010
\item
  License: CC BY 4.0
\end{itemize}

\end{Form}

\end{document}
