% Pandoc LaTeX template for checklists with basic PDF forms
\documentclass[11pt]{article}
\usepackage[margin=1in]{geometry}
\usepackage{hyperref}
\usepackage{amssymb}
% Optional AcroTeX eForms for richer widgets
\usepackage{eforms}
\hypersetup{colorlinks=true,linkcolor=blue,urlcolor=blue}
\usepackage{enumitem}
\setlist[itemize]{left=1.2em}
\setlist{noitemsep}
\def\tightlist{}

% Abstraction layer for form widgets
  % eForms uses \checkBox and \textField
  \newcommand{\mkCheckBox}[2][]{\checkBox[#1]{#2}}
  \newcommand{\mkTextField}[2][]{\textField[#1]{#2}}

% Enable form environment for interactive fields
% Usage in content requires raw LaTeX or a pandoc Lua filter to insert \CheckBox/\TextField
\begin{document}

\begin{center}
{\LARGE }\\[4pt]
\normalsize 
\end{center}
\vspace{1em}

% Begin PDF form region
\begin{Form}

\section{CARE Guidelines Checklist}\label{care-guidelines-checklist}

\begin{quote}
Scope: The CARE guidelines provide a framework for authors to follow
when reporting on cases. The checklist is designed to correspond with
the key components of a case report and to capture essential clinical
information.

Reference: See \texttt{source/archetypes/care-2013.yml} for canonical
link and provenance.
\end{quote}

\subsection{Instructions}\label{instructions}

\begin{itemize}
\tightlist
\item
  Use task list items for checklist boxes; these become interactive
  checkboxes in PDF.
\item
  Use a span with class \texttt{.textfield} for free‑text fields.
\end{itemize}

\subsection{Items}\label{items}

\begin{itemize}
\tightlist
\item[$\square$]
  \textbf{Title:} The title should include the words ``case report'' and
  identify the primary focus of the report.
\item[$\square$]
  \textbf{Key Words:} Two to five keywords that identify the diagnoses
  or interventions covered in the case report.
\item[$\square$]
  \textbf{Abstract:} A structured or unstructured summary that includes
  an introduction to the case's uniqueness, the patient's main symptoms
  and clinical findings, the primary diagnoses and interventions, and
  the main ``take-away'' lessons.
\item[$\square$]
  \textbf{Introduction:} A brief summary of why the case is unique,
  potentially with references to existing medical literature.
\item[$\square$]
  \textbf{Patient Information:} This section should include
  de-identified demographic and other specific information about the
  patient, their main concerns and symptoms, and their medical, family,
  and psychosocial history, including relevant genetic information and
  past interventions with their outcomes.
\item[$\square$]
  \textbf{Clinical Findings:} A description of the relevant physical
  examination and other significant clinical findings.
\item[$\square$]
  \textbf{Timeline:} A timeline that organizes important historical and
  current information from the episode of care, which can be presented
  as a figure or table.
\item[$\square$]
  \textbf{Diagnostic Assessment:} This includes the diagnostic methods
  used (such as physical examination, laboratory testing, imaging, and
  surveys), any diagnostic challenges, the final diagnosis (including
  other diagnoses that were considered), and prognostic characteristics
  where applicable.
\item[$\square$]
  \textbf{Therapeutic Intervention:} Details on the types of
  interventions (e.g., pharmacologic, surgical, preventive), how they
  were administered (dosage, strength, duration), and any changes made
  to the interventions with explanations.
\item[$\square$]
  \textbf{Follow-up and Outcomes:} This should cover clinician- and
  patient-assessed outcomes, important follow-up diagnostic and other
  test results, intervention adherence and tolerability, and any adverse
  or unanticipated events.
\item[$\square$]
  \textbf{Discussion:} A discussion of the strengths and limitations of
  the approach to the case, a review of relevant medical literature, the
  rationale for the conclusions drawn, and the primary ``take-away''
  lessons from the report.
\item[$\square$]
  \textbf{Patient Perspective:} The patient should have the opportunity
  to share their perspective on the treatment(s) they received.
\item[$\square$]
  \textbf{Informed Consent:} Confirmation that the patient gave informed
  consent for the report.
\end{itemize}

{Notes}

\subsection{Provenance}\label{provenance}

\begin{itemize}
\tightlist
\item
  Source: See sidecar metadata in
  \texttt{source/archetypes/care-2013.yml}
\item
  Version: 2013
\item
  License: CC-BY-NC-4.0
\end{itemize}

\end{Form}

\end{document}
