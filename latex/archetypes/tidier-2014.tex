% Pandoc LaTeX template for checklists with basic PDF forms
\documentclass[11pt]{article}
\usepackage[margin=1in]{geometry}
\usepackage{hyperref}
\usepackage{amssymb}
% Optional AcroTeX eForms for richer widgets
\usepackage{eforms}
\hypersetup{colorlinks=true,linkcolor=blue,urlcolor=blue}
\usepackage{enumitem}
\setlist[itemize]{left=1.2em}
\setlist{noitemsep}
\def\tightlist{}

% Abstraction layer for form widgets
  % eForms uses \checkBox and \textField
  \newcommand{\mkCheckBox}[2][]{\checkBox[#1]{#2}}
  \newcommand{\mkTextField}[2][]{\textField[#1]{#2}}

% Enable form environment for interactive fields
% Usage in content requires raw LaTeX or a pandoc Lua filter to insert \CheckBox/\TextField
\begin{document}

\begin{center}
{\LARGE }\\[4pt]
\normalsize 
\end{center}
\vspace{1em}

% Begin PDF form region
\begin{Form}

\section{TIDieR Checklist}\label{tidier-checklist}

\begin{quote}
Scope: Template for Intervention Description and Replication.

Reference: See \texttt{source/archetypes/tidier-2014.yml} for canonical
link and provenance.
\end{quote}

\subsection{Instructions}\label{instructions}

\begin{itemize}
\tightlist
\item
  Use the boxes to confirm each reporting item.
\item
  Add reviewer notes under each section as needed.
\end{itemize}

\subsection{Brief name}\label{brief-name}

\begin{itemize}
\tightlist
\item[$\square$]
  \textbf{1. Brief name:} Provide the name or a phrase that describes
  the intervention.
\end{itemize}

\subsection{Why}\label{why}

\begin{itemize}
\tightlist
\item[$\square$]
  \textbf{2. Why:} Describe any rationale, theory, or goal of the
  elements essential to the intervention.
\end{itemize}

\subsection{What}\label{what}

\begin{itemize}
\tightlist
\item[$\square$]
  \textbf{3. What (materials):} Describe any physical or informational
  materials used in the intervention, including those provided to
  participants or used in intervention delivery or training of
  intervention providers.
\item[$\square$]
  \textbf{4. What (procedures):} Describe each of the procedures,
  activities, and/or processes used in the intervention, including any
  enabling or support activities.
\end{itemize}

\subsection{Who provided}\label{who-provided}

\begin{itemize}
\tightlist
\item[$\square$]
  \textbf{5. Who provided:} For each category of intervention provider
  (e.g., psychologist, nursing assistant), describe their expertise,
  background, and any specific training given.
\end{itemize}

\subsection{How}\label{how}

\begin{itemize}
\tightlist
\item[$\square$]
  \textbf{6. How:} Describe the mode(s) of delivery (e.g., face-to-face,
  internet) and the setting (e.g., home, clinic) for each component of
  the intervention.
\end{itemize}

\subsection{Where}\label{where}

\begin{itemize}
\tightlist
\item[$\square$]
  \textbf{7. Where:} Describe the location(s) where the intervention was
  delivered, including any necessary infrastructure or relevant
  features.
\end{itemize}

\subsection{When and how much}\label{when-and-how-much}

\begin{itemize}
\tightlist
\item[$\square$]
  \textbf{8. When and how much:} Describe the number of times the
  intervention was delivered and over what period of time, including the
  number of sessions, their schedule, and their duration, intensity, or
  dose.
\end{itemize}

\subsection{Tailoring}\label{tailoring}

\begin{itemize}
\tightlist
\item[$\square$]
  \textbf{9. Tailoring:} If the intervention was planned to be
  personalized, titrated, or adapted, then describe what, why, when, and
  how.
\end{itemize}

\subsection{Modifications}\label{modifications}

\begin{itemize}
\tightlist
\item[$\square$]
  \textbf{10. Modifications:} If the intervention was modified during
  the course of the study, describe the changes (what, why, when, and
  how).
\end{itemize}

\subsection{How well}\label{how-well}

\begin{itemize}
\tightlist
\item[$\square$]
  \textbf{11. How well (planned):} If intervention adherence or fidelity
  was assessed, describe how and by whom, and if any strategies were
  used to maintain or improve fidelity.
\item[$\square$]
  \textbf{12. How well (actual):} If intervention adherence or fidelity
  was assessed, describe the extent to which the intervention was
  delivered as planned.
\end{itemize}

\subsubsection{Notes}\label{notes}

{Reviewer notes}

\subsection{Provenance}\label{provenance}

\begin{itemize}
\tightlist
\item
  Source: See sidecar metadata in
  \texttt{source/archetypes/tidier-2014.yml}
\item
  Version: 2014
\item
  License: CC-BY-NC
\end{itemize}

\end{Form}

\end{document}
