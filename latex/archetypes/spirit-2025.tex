% Pandoc LaTeX template for checklists with basic PDF forms
\documentclass[11pt]{article}
\usepackage[margin=1in]{geometry}
\usepackage{hyperref}
\usepackage{amssymb}
% Optional AcroTeX eForms for richer widgets
\usepackage{eforms}
\hypersetup{colorlinks=true,linkcolor=blue,urlcolor=blue}
\usepackage{enumitem}
\setlist[itemize]{left=1.2em}
\setlist{noitemsep}
\def\tightlist{}

% Abstraction layer for form widgets
  % eForms uses \checkBox and \textField
  \newcommand{\mkCheckBox}[2][]{\checkBox[#1]{#2}}
  \newcommand{\mkTextField}[2][]{\textField[#1]{#2}}

% Enable form environment for interactive fields
% Usage in content requires raw LaTeX or a pandoc Lua filter to insert \CheckBox/\TextField
\begin{document}

\begin{center}
{\LARGE }\\[4pt]
\normalsize 
\end{center}
\vspace{1em}

% Begin PDF form region
\begin{Form}

\section{SPIRIT 2025 Checklist}\label{spirit-2025-checklist}

\begin{quote}
Scope: Standard Protocol Items: Recommendations for Interventional
Trials.

Reference: See \texttt{source/archetypes/spirit-2025.yml} for canonical
link and provenance.
\end{quote}

\subsection{Instructions}\label{instructions}

\begin{itemize}
\tightlist
\item
  Use the boxes to confirm each reporting item.
\item
  Add reviewer notes under each section as needed.
\end{itemize}

\subsection{Administrative
information}\label{administrative-information}

\begin{itemize}
\tightlist
\item[$\square$]
  \textbf{1. Title:} Descriptive title that identifies the study as a
  randomised trial, the interventions, the trial acronym, and the SPIRIT
  item this protocol is based on.
\item[$\square$]
  \textbf{2. Trial registration:} Trial identifier and registry name. If
  not yet registered, name of intended registry.
\item[$\square$]
  \textbf{3. Protocol version:} Date and version identifier.
\item[$\square$]
  \textbf{4. Funding:} Sources and types of financial, material, and
  other support.
\item[$\square$]
  \textbf{5. Roles and responsibilities:} Names, affiliations, and roles
  of protocol contributors.
\end{itemize}

\subsection{Introduction}\label{introduction}

\begin{itemize}
\tightlist
\item[$\square$]
  \textbf{6. Background and rationale:} Description of research question
  and justification for undertaking the trial, including summary of
  relevant studies.
\item[$\square$]
  \textbf{7. Objectives:} Specific objectives or hypotheses.
\end{itemize}

\subsection{Methods: Participants, interventions, and
outcomes}\label{methods-participants-interventions-and-outcomes}

\begin{itemize}
\tightlist
\item[$\square$]
  \textbf{8. Trial design:} Description of trial design including type
  of trial, allocation ratio, and framework.
\item[$\square$]
  \textbf{9. Study setting:} Description of study settings.
\item[$\square$]
  \textbf{10. Eligibility criteria:} Inclusion and exclusion criteria
  for participants.
\item[$\square$]
  \textbf{11. Interventions:} Interventions for each group with
  sufficient detail to allow replication.
\item[$\square$]
  \textbf{12. Outcomes:} Primary, secondary, and other outcomes.
\end{itemize}

\subsection{Methods: Assignment of interventions (for controlled
trials)}\label{methods-assignment-of-interventions-for-controlled-trials}

\begin{itemize}
\tightlist
\item[$\square$]
  \textbf{13. Allocation:} Sequence generation, concealment mechanism,
  and implementation.
\item[$\square$]
  \textbf{14. Blinding (masking):} Who will be blinded and how.
\end{itemize}

\subsection{Methods: Data collection, management, and
analysis}\label{methods-data-collection-management-and-analysis}

\begin{itemize}
\tightlist
\item[$\square$]
  \textbf{15. Data collection methods:} Plans for assessment and
  collection of outcome, baseline, and other trial data.
\item[$\square$]
  \textbf{16. Data management:} Plans for data entry, coding, security,
  and storage.
\item[$\square$]
  \textbf{17. Statistical methods:} Statistical methods for analysing
  primary and secondary outcomes.
\end{itemize}

\subsection{Methods: Monitoring}\label{methods-monitoring}

\begin{itemize}
\tightlist
\item[$\square$]
  \textbf{18. Data monitoring:} Plans for data monitoring.
\item[$\square$]
  \textbf{19. Harms:} Plans for collecting, assessing, reporting, and
  managing solicited and spontaneously reported adverse events and other
  unintended effects of trial interventions or trial conduct.
\item[$\square$]
  \textbf{20. Auditing:} Frequency and procedures for auditing trial
  conduct.
\end{itemize}

\subsection{Ethics and dissemination}\label{ethics-and-dissemination}

\begin{itemize}
\tightlist
\item[$\square$]
  \textbf{21. Research ethics approval:} Plans for seeking research
  ethics committee/institutional review board approval.
\item[$\square$]
  \textbf{22. Protocol amendments:} Plans for communicating important
  protocol modifications to relevant parties.
\item[$\square$]
  \textbf{23. Consent or assent:} Who will obtain informed consent or
  assent from potential trial participants or authorized surrogates, and
  how.
\item[$\square$]
  \textbf{24. Confidentiality:} How personal information about potential
  and enrolled participants will be collected, shared, and maintained in
  order to protect confidentiality before, during, and after the trial.
\item[$\square$]
  \textbf{25. Declaration of interests:} Financial and other competing
  interests for principal investigators for the overall trial and each
  study site.
\item[$\square$]
  \textbf{26. Access to data:} Statement of who will have access to the
  final trial dataset, and disclosure of contractual agreements that
  limit such access for investigators.
\item[$\square$]
  \textbf{27. Ancillary and post-trial care:} Provisions, if any, for
  ancillary and post-trial care, and for compensation to those who
  suffer harm from trial participation.
\item[$\square$]
  \textbf{28. Dissemination policy:} Plans for investigators and
  sponsors to authors and other stakeholders to share trial results.
\item[$\square$]
  \textbf{29. Authorship eligibility:} Guidelines for authorship
  eligibility for trial publications.
\item[$\square$]
  \textbf{30. Reproducibility:} Plans for sharing of original data and
  statistical code.
\end{itemize}

\subsection{Appendices}\label{appendices}

\begin{itemize}
\tightlist
\item[$\square$]
  \textbf{31. Informed consent materials:} Model consent form and other
  related documentation given to participants and authorized surrogates.
\item[$\square$]
  \textbf{32. Biological specimens:} Plans for collection, laboratory
  evaluation, and storage of biological specimens for genetic or
  molecular analysis in the current trial and for future use in
  ancillary studies, if applicable.
\end{itemize}

{Notes}

\subsection{Provenance}\label{provenance}

\begin{itemize}
\tightlist
\item
  Source: See sidecar metadata in
  \texttt{source/archetypes/spirit-2025.yml}
\item
  Version: 2025
\item
  License: CC-BY-4.0
\end{itemize}

\end{Form}

\end{document}
