% Pandoc LaTeX template for checklists with basic PDF forms
\documentclass[11pt]{article}
\usepackage[margin=1in]{geometry}
\usepackage{hyperref}
\usepackage{amssymb}
% Optional AcroTeX eForms for richer widgets
\usepackage{eforms}
\hypersetup{colorlinks=true,linkcolor=blue,urlcolor=blue}
\usepackage{enumitem}
\setlist[itemize]{left=1.2em}
\setlist{noitemsep}
\def\tightlist{}

% Abstraction layer for form widgets
  % eForms uses \checkBox and \textField
  \newcommand{\mkCheckBox}[2][]{\checkBox[#1]{#2}}
  \newcommand{\mkTextField}[2][]{\textField[#1]{#2}}

% Enable form environment for interactive fields
% Usage in content requires raw LaTeX or a pandoc Lua filter to insert \CheckBox/\TextField
\begin{document}

\begin{center}
{\LARGE }\\[4pt]
\normalsize 
\end{center}
\vspace{1em}

% Begin PDF form region
\begin{Form}

\section{ARRIVE 2.0 Checklist}\label{arrive-2.0-checklist}

\begin{quote}
Scope: The ARRIVE 2.0 guidelines are a comprehensive checklist for
reporting animal research. These guidelines are designed to improve the
transparency and quality of scientific publications.

Reference: See \texttt{source/archetypes/arrive-2020.yml} for canonical
link and provenance.
\end{quote}

\subsection{Instructions}\label{instructions}

\begin{itemize}
\tightlist
\item
  Use task list items for checklist boxes; these become interactive
  checkboxes in PDF.
\item
  Use a span with class \texttt{.textfield} for free‑text fields.
\end{itemize}

\subsection{The ARRIVE Essential 10}\label{the-arrive-essential-10}

\begin{itemize}
\tightlist
\item[$\square$]
  \textbf{Study Design:} A brief description of the study design for
  each experiment, including the groups being compared (with control
  groups) and the experimental unit (e.g., a single animal or a cage of
  animals).
\item[$\square$]
  \textbf{Sample Size:} The exact number of experimental units in each
  group and the total number for each experiment, along with the total
  number of animals used. The method for determining the sample size
  should also be explained.
\item[$\square$]
  \textbf{Inclusion and Exclusion Criteria:} A description of the
  criteria for including or excluding animals or data points during the
  experiment and analysis.
\item[$\square$]
  \textbf{Randomization:} A statement on whether randomization was used
  to assign experimental units to groups and a description of the method
  used.
\item[$\square$]
  \textbf{Blinding:} A description of who was blinded during the
  different stages of the experiment (e.g., during allocation, outcome
  assessment, and data analysis).
\item[$\square$]
  \textbf{Outcome Measures:} A clear definition of all assessed outcome
  measures. For studies testing a hypothesis, the primary outcome
  measure should be specified.
\item[$\square$]
  \textbf{Statistical Methods:} Detailed information about the
  statistical methods used for each analysis, including the software
  used.
\item[$\square$]
  \textbf{Experimental Animals:} Species-appropriate details of the
  animals used, such as species, strain, sex, and age.
\item[$\square$]
  \textbf{Experimental Procedures:} A thorough description of the
  procedures for each experimental group, detailed enough for
  replication by others.
\item[$\square$]
  \textbf{Results:} For each experiment, a report of summary and
  descriptive statistics for each group, including a measure of
  variability.
\end{itemize}

\subsection{The Recommended Set}\label{the-recommended-set}

\begin{itemize}
\tightlist
\item[$\square$]
  \textbf{Abstract:} A summary of the study's background, objectives,
  methods, principal findings, and conclusions.
\item[$\square$]
  \textbf{Background:} An explanation of the scientific context and the
  experimental approach.
\item[$\square$]
  \textbf{Objectives:} A clear statement of the research objectives and
  any specific hypotheses being tested.
\item[$\square$]
  \textbf{Ethical Statement:} The name of the ethical review committee
  that approved the study and any relevant license or protocol numbers.
\item[$\square$]
  \textbf{Housing and Husbandry:} Details of the housing and husbandry
  conditions for the animals.
\item[$\square$]
  \textbf{Animal Care and Monitoring:} A description of any
  interventions to reduce pain and suffering, as well as any adverse
  events and humane endpoints.
\item[$\square$]
  \textbf{Interpretation/Scientific Implications:} An interpretation of
  the results in the context of the study's objectives and existing
  literature.
\item[$\square$]
  \textbf{Generalizability/Translation:} A discussion of the
  generalizability of the findings.
\item[$\square$]
  \textbf{Protocol Registration:} A statement indicating if a protocol
  was prepared before the study and where it was registered.
\item[$\square$]
  \textbf{Data Access:} A statement on where the study data can be
  accessed.
\item[$\square$]
  \textbf{Declaration of Interest:} A declaration of any potential
  conflicts of interest.
\item[$\square$]
  \textbf{Funding:} Information on all funding sources and the role of
  the funder(s).
\end{itemize}

{Notes}

\subsection{Provenance}\label{provenance}

\begin{itemize}
\tightlist
\item
  Source: See sidecar metadata in
  \texttt{source/archetypes/arrive-2020.yml}
\item
  Version: 2.0
\item
  License: CC-BY-4.0
\end{itemize}

\end{Form}

\end{document}
