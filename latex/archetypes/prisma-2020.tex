% Pandoc LaTeX template for checklists with basic PDF forms
\documentclass[11pt]{article}
\usepackage[margin=1in]{geometry}
\usepackage{hyperref}
\usepackage{amssymb}
% Optional AcroTeX eForms for richer widgets
\usepackage{eforms}
\hypersetup{colorlinks=true,linkcolor=blue,urlcolor=blue}
\usepackage{enumitem}
\setlist[itemize]{left=1.2em}
\setlist{noitemsep}
\def\tightlist{}

% Abstraction layer for form widgets
  % eForms uses \checkBox and \textField
  \newcommand{\mkCheckBox}[2][]{\checkBox[#1]{#2}}
  \newcommand{\mkTextField}[2][]{\textField[#1]{#2}}

% Enable form environment for interactive fields
% Usage in content requires raw LaTeX or a pandoc Lua filter to insert \CheckBox/\TextField
\begin{document}

\begin{center}
{\LARGE }\\[4pt]
\normalsize 
\end{center}
\vspace{1em}

% Begin PDF form region
\begin{Form}

\section{PRISMA 2020 Checklist}\label{prisma-2020-checklist}

\begin{quote}
Scope: Preferred Reporting Items for Systematic Reviews and
Meta-Analyses.

Reference: See \texttt{source/archetypes/prisma-2020.yml} for canonical
link and provenance.
\end{quote}

\subsection{Instructions}\label{instructions}

\begin{itemize}
\tightlist
\item
  Use the boxes to confirm each reporting item.
\item
  Add reviewer notes under each section as needed.
\end{itemize}

\subsection{Title}\label{title}

\begin{itemize}
\tightlist
\item[$\square$]
  \textbf{1. Title:} Identify the report as a systematic review.
\end{itemize}

\subsection{Abstract}\label{abstract}

\begin{itemize}
\tightlist
\item[$\square$]
  \textbf{2. Abstract:} See the PRISMA 2020 for Abstracts checklist.
\end{itemize}

\subsection{Introduction}\label{introduction}

\begin{itemize}
\tightlist
\item[$\square$]
  \textbf{3. Rationale:} Describe the rationale for the review in the
  context of existing knowledge.
\item[$\square$]
  \textbf{4. Objectives:} Provide an explicit statement of the main
  objective(s) or question(s) the review addresses.
\end{itemize}

\subsection{Methods}\label{methods}

\begin{itemize}
\tightlist
\item[$\square$]
  \textbf{5. Eligibility criteria:} Specify the inclusion and exclusion
  criteria for the review and how studies were grouped for the
  syntheses.
\item[$\square$]
  \textbf{6. Information sources:} Specify all databases, registers,
  websites, organisations, reference lists and other sources searched or
  consulted to identify studies. Specify the date when each source was
  last searched or consulted.
\item[$\square$]
  \textbf{7. Search strategy:} Present the full search strategies for
  all databases, registers, and websites, including any filters and
  limits used.
\item[$\square$]
  \textbf{8. Selection process:} Specify the methods used to decide
  whether a study met the inclusion criteria of the review, including
  how many reviewers screened each record and each report retrieved,
  whether they worked independently, and if applicable, details of
  automation tools used in the process.
\item[$\square$]
  \textbf{9. Data collection process:} Specify the methods used to
  collect data from reports, including how many reviewers collected data
  from each report, whether they worked independently, any processes for
  obtaining or confirming data from study investigators, and if
  applicable, details of automation tools used in the process.
\item[$\square$]
  \textbf{10. Data items:} List and define all outcomes for which data
  were sought. Specify whether and how results were sought for each
  outcome, by who, and how they were categorised.
\item[$\square$]
  \textbf{11. Study risk of bias assessment:} Specify the methods used
  to assess risk of bias in the included studies, including details of
  the tool(s) used, how many reviewers assessed each study and whether
  they worked independently, and if applicable, details of automation
  tools used in the process.
\item[$\square$]
  \textbf{12. Effect measures:} Specify the effect measure(s) (e.g.,
  risk ratio, mean difference) used in the synthesis or presentation of
  results for each outcome.
\item[$\square$]
  \textbf{13. Synthesis methods:} Describe the processes used to decide
  which studies were eligible for each synthesis (e.g., tabulating the
  study intervention characteristics and comparing them against the
  planned groups for each synthesis (item 5)).
\item[$\square$]
  \textbf{14. Reporting bias assessment:} Describe any methods used to
  assess the risk of bias due to missing results in a synthesis (arising
  from reporting biases).
\item[$\square$]
  \textbf{15. Certainty assessment:} Describe any methods used to assess
  certainty (or confidence) in the body of evidence for an outcome.
\end{itemize}

\subsection{Results}\label{results}

\begin{itemize}
\tightlist
\item[$\square$]
  \textbf{16. Study selection:} Describe the results of the search and
  selection process, from the number of records identified in the search
  to the number of studies included in the review, ideally using a flow
  diagram.
\item[$\square$]
  \textbf{17. Study characteristics:} Cite each included study and
  present its characteristics.
\item[$\square$]
  \textbf{18. Risk of bias in studies:} Present assessments of risk of
  bias for each included study.
\item[$\square$]
  \textbf{19. Results of individual studies:} For all outcomes, present,
  for each study: (a) summary statistics for each group (where
  appropriate) and (b) an effect estimate and its precision (e.g.,
  confidence/credible interval), ideally using structured tables or
  plots.
\item[$\square$]
  \textbf{20. Results of syntheses:} For each synthesis, present results
  for all outcomes that were assessed, and for each outcome, present a
  summary of findings and, if meta-analysis was done, an effect estimate
  and its precision.
\item[$\square$]
  \textbf{21. Reporting biases:} Present assessments of risk of bias due
  to missing results (arising from reporting biases) for each synthesis
  assessed.
\item[$\square$]
  \textbf{22. Certainty of evidence:} Present assessments of certainty
  (or confidence) in the body of evidence for each outcome assessed.
\end{itemize}

\subsection{Discussion}\label{discussion}

\begin{itemize}
\tightlist
\item[$\square$]
  \textbf{23. Discussion:} (a) Provide a general interpretation of the
  results in the context of other evidence. (b) Discuss any limitations
  of the evidence included in the review. (c) Discuss any limitations of
  the review processes used. (d) Discuss the implications of the results
  for practice, policy, and future research.
\end{itemize}

\subsection{Other Information}\label{other-information}

\begin{itemize}
\tightlist
\item[$\square$]
  \textbf{24. Registration and protocol:} Provide registration
  information for the review, including register name and registration
  number, or state that the review was not registered.
\item[$\square$]
  \textbf{25. Support:} Specify the sources of financial or
  non-financial support for the review, and the role of the funders or
  sponsors in the review.
\item[$\square$]
  \textbf{26. Competing interests:} Declare any competing interests of
  review authors.
\item[$\square$]
  \textbf{27. Availability of data, code and other materials:} Report
  which of the following are publicly available and where they can be
  found: template data collection forms; data extracted from included
  studies; data used for all analyses; analytic code; any other
  materials used in the review.
\end{itemize}

\subsubsection{Notes}\label{notes}

{Reviewer notes}

\subsection{Provenance}\label{provenance}

\begin{itemize}
\tightlist
\item
  Source: See sidecar metadata in
  \texttt{source/archetypes/prisma-2020.yml}
\item
  Version: 2020
\item
  License: CC BY 4.0
\end{itemize}

\end{Form}

\end{document}
