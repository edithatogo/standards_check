% Pandoc LaTeX template for checklists with basic PDF forms
\documentclass[11pt]{article}
\usepackage[margin=1in]{geometry}
\usepackage{hyperref}
\usepackage{amssymb}
% Optional AcroTeX eForms for richer widgets
\usepackage{eforms}
\hypersetup{colorlinks=true,linkcolor=blue,urlcolor=blue}
\usepackage{enumitem}
\setlist[itemize]{left=1.2em}
\setlist{noitemsep}
\def\tightlist{}

% Abstraction layer for form widgets
  % eForms uses \checkBox and \textField
  \newcommand{\mkCheckBox}[2][]{\checkBox[#1]{#2}}
  \newcommand{\mkTextField}[2][]{\textField[#1]{#2}}

% Enable form environment for interactive fields
% Usage in content requires raw LaTeX or a pandoc Lua filter to insert \CheckBox/\TextField
\begin{document}

\begin{center}
{\LARGE PRISMA-ScR Checklist}\\[4pt]
\normalsize Date: 2018-10-02
\end{center}
\vspace{1em}

% Begin PDF form region
\begin{Form}

\section{PRISMA-ScR Checklist}\label{prisma-scr-checklist}

\begin{quote}
Scope: Preferred Reporting Items for Systematic Reviews and
Meta-Analyses extension for Scoping Reviews.

Reference: See \texttt{source/variants/prisma-scr.yml} for canonical
link and provenance.
\end{quote}

\subsection{Instructions}\label{instructions}

\begin{itemize}
\tightlist
\item
  Use the boxes to confirm each reporting item.
\item
  Add reviewer notes under each section as needed.
\end{itemize}

\subsection{Title}\label{title}

\begin{itemize}
\tightlist
\item[$\square$]
  \textbf{1. Title:} Identify the report as a scoping review.
\end{itemize}

\subsection{Abstract}\label{abstract}

\begin{itemize}
\tightlist
\item[$\square$]
  \textbf{2. Structured summary:} Provide a structured summary
  including, as applicable: background, objectives, eligibility
  criteria, sources of evidence, charting methods, results, and
  conclusions that relate to the review questions and objectives.
\end{itemize}

\subsection{Introduction}\label{introduction}

\begin{itemize}
\tightlist
\item[$\square$]
  \textbf{3. Rationale:} Describe the rationale for the review in the
  context of existing knowledge and explain why a scoping review is a
  suitable approach.
\item[$\square$]
  \textbf{4. Objectives:} State the review questions and objectives
  clearly, referencing their key elements (e.g., population, concepts,
  context).
\end{itemize}

\subsection{Methods}\label{methods}

\begin{itemize}
\tightlist
\item[$\square$]
  \textbf{5. Protocol and registration:} Indicate if a review protocol
  exists, and if so, where it can be accessed (e.g., a web address) and
  provide registration information if available.
\item[$\square$]
  \textbf{6. Eligibility criteria:} Specify the characteristics of the
  sources of evidence used as eligibility criteria (e.g., years
  considered, language, publication status) and provide a rationale.
\item[$\square$]
  \textbf{7. Information sources:} Describe all information sources used
  in the search (e.g., databases with dates of coverage, contact with
  authors) and the date of the most recent search.
\item[$\square$]
  \textbf{8. Search:} Present the full electronic search strategy for at
  least one database, including any limits used, so it can be repeated.
\item[$\square$]
  \textbf{9. Selection of sources of evidence:} State the process for
  selecting sources of evidence (i.e., screening and eligibility) for
  inclusion in the scoping review.
\item[$\square$]
  \textbf{10. Data charting:} Describe the methods for charting data
  from the included sources of evidence and any processes for obtaining
  and confirming data from investigators.
\item[$\square$]
  \textbf{11. Data items:} List and define all variables for which data
  were sought and any assumptions and simplifications made.
\item[$\square$]
  \textbf{12. Critical appraisal of individual sources of evidence
  (Optional):} If done, describe the methods used for critical appraisal
  of individual sources of evidence and how this information was used in
  the synthesis of results.
\end{itemize}

\subsection{Results}\label{results}

\begin{itemize}
\tightlist
\item[$\square$]
  \textbf{13. Selection of sources of evidence:} Give the number of
  sources of evidence screened, assessed for eligibility, and included
  in the review, with reasons for exclusions at each stage, ideally with
  a flow diagram.
\item[$\square$]
  \textbf{14. Characteristics of sources of evidence:} For each source
  of evidence, present its characteristics for which data were charted
  and any critical appraisal results.
\item[$\square$]
  \textbf{15. Results of individual sources of evidence:} For each
  included source of evidence, present the relevant data that were
  charted.
\item[$\square$]
  \textbf{16. Synthesis of results:} Summarize and synthesize the
  results, and present them in a logical and structured manner that
  aligns with the review's objectives and questions.
\end{itemize}

\subsection{Discussion}\label{discussion}

\begin{itemize}
\tightlist
\item[$\square$]
  \textbf{17. Summary of evidence:} Summarize the main results,
  including a discussion of how they relate to the review's questions
  and objectives.
\item[$\square$]
  \textbf{18. Limitations:} Discuss the limitations of the scoping
  review process.
\item[$\square$]
  \textbf{19. Conclusions:} Provide a general interpretation of the
  results in the context of the review questions and objectives, as well
  as potential implications and/or next steps.
\end{itemize}

\subsection{Funding}\label{funding}

\begin{itemize}
\tightlist
\item[$\square$]
  \textbf{20. Funding:} Describe the sources of funding for the scoping
  review and the role of the funders.
\end{itemize}

\subsection{Optional Items}\label{optional-items}

\begin{itemize}
\tightlist
\item[$\square$]
  \textbf{21. Critical appraisal within sources of evidence (Optional):}
  If done, present the results of any critical appraisal that was
  conducted within the sources of evidence.
\item[$\square$]
  \textbf{22. Funding of included sources of evidence:} Describe sources
  of funding for the included sources of evidence.
\end{itemize}

\subsubsection{Notes}\label{notes}

{Reviewer notes}

\subsection{Provenance}\label{provenance}

\begin{itemize}
\tightlist
\item
  Source: See sidecar metadata in
  \texttt{source/variants/prisma-scr.yml}
\item
  Version: 2018
\item
  License: CC-BY-4.0
\end{itemize}

\end{Form}

\end{document}
