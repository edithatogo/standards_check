% Pandoc LaTeX template for checklists with basic PDF forms
\documentclass[11pt]{article}
\usepackage[margin=1in]{geometry}
\usepackage{hyperref}
\usepackage{amssymb}
% Optional AcroTeX eForms for richer widgets
\usepackage{eforms}
\hypersetup{colorlinks=true,linkcolor=blue,urlcolor=blue}
\usepackage{enumitem}
\setlist[itemize]{left=1.2em}
\setlist{noitemsep}
\def\tightlist{}

% Abstraction layer for form widgets
  % eForms uses \checkBox and \textField
  \newcommand{\mkCheckBox}[2][]{\checkBox[#1]{#2}}
  \newcommand{\mkTextField}[2][]{\textField[#1]{#2}}

% Enable form environment for interactive fields
% Usage in content requires raw LaTeX or a pandoc Lua filter to insert \CheckBox/\TextField
\begin{document}

\begin{center}
{\LARGE CONSORT for N-of-1 Trials}\\[4pt]
\normalsize Date: 2015-04-29
\end{center}
\vspace{1em}

% Begin PDF form region
\begin{Form}

\section{CONSORT for N-of-1 Trials
Checklist}\label{consort-for-n-of-1-trials-checklist}

\begin{quote}
Scope: Preferred Reporting Items for N-of-1 trials.

Reference: See \texttt{source/variants/consort-n-of-1.yml} for canonical
link and provenance.
\end{quote}

\subsection{Instructions}\label{instructions}

\begin{itemize}
\tightlist
\item
  Use the boxes to confirm each reporting item.
\item
  Add reviewer notes under each section as needed.
\end{itemize}

\subsection{Checklist Items}\label{checklist-items}

\begin{itemize}
\tightlist
\item[$\square$]
  \textbf{Title and abstract}

  \begin{itemize}
  \tightlist
  \item[$\square$]
    1a. Identification as an N-of-1 trial in the title
  \item[$\square$]
    1b. Structured summary of trial design, methods, results, and
    conclusions
  \end{itemize}
\item[$\square$]
  \textbf{Introduction}

  \begin{itemize}
  \tightlist
  \item[$\square$]
    2a. Scientific background and explanation of rationale
  \item[$\square$]
    2b. Specific objectives or hypotheses
  \end{itemize}
\item[$\square$]
  \textbf{Methods}

  \begin{itemize}
  \tightlist
  \item[$\square$]
    3a. Description of trial design (including allocation ratio, if
    applicable)
  \item[$\square$]
    3b. Important changes to methods after trial commencement (such as
    eligibility criteria), with reasons
  \item[$\square$]
    4a. Eligibility criteria for participants
  \item[$\square$]
    4b. Settings and locations where the data were collected
  \item[$\square$]
    5. The interventions for each group with sufficient details to allow
    replication, including how and when they were actually administered
  \item[$\square$]
    6a. Completely defined pre-specified primary and secondary outcome
    measures, including how and when they were assessed
  \item[$\square$]
    6b. Any changes to trial outcomes after the trial commenced, with
    reasons
  \item[$\square$]
    7a. How sample size was determined
  \item[$\square$]
    7b. When applicable, explanation of any interim analyses and
    stopping guidelines
  \item[$\square$]
    8a. Method used to generate the random allocation sequence
  \item[$\square$]
    8b. Type of randomisation; details of any restriction (e.g.,
    blocking and block size)
  \item[$\square$]
    9. Mechanism used to implement the random allocation sequence (e.g.,
    central telephone; web-based), describing any steps taken to conceal
    the sequence until interventions were assigned
  \item[$\square$]
    10. Who generated the random allocation sequence, who enrolled
    participants, and who assigned participants to interventions
  \item[$\square$]
    11a. If done, who was blinded after assignment to interventions
    (e.g., participants, care providers, those assessing outcomes) and
    how
  \item[$\square$]
    11b. If relevant, description of the similarity of interventions
  \item[$\square$]
    12a. Statistical methods used to compare groups for primary and
    secondary outcomes
  \item[$\square$]
    12b. Methods for additional analyses, such as subgroup analyses and
    adjusted analyses
  \end{itemize}
\item[$\square$]
  \textbf{Results}

  \begin{itemize}
  \tightlist
  \item[$\square$]
    13a. For each group, the numbers of participants who were randomly
    assigned, received intended treatment, and were analysed for the
    primary outcome
  \item[$\square$]
    13b. For each group, losses and exclusions after randomisation,
    together with reasons
  \item[$\square$]
    14a. Dates defining the periods of recruitment and follow-up
  \item[$\square$]
    14b. Why the trial ended or was stopped
  \item[$\square$]
    15. A table showing baseline demographic and clinical
    characteristics for each group
  \item[$\square$]
    16. For each group, number of participants (denominator) included in
    each analysis and whether the analysis was by original assigned
    groups
  \item[$\square$]
    17a. For each primary and secondary outcome, results for each group,
    and the estimated effect size and its precision (e.g., 95\%
    confidence interval)
  \item[$\square$]
    17b. For binary outcomes, presentation of both absolute and relative
    effect sizes is recommended
  \item[$\square$]
    18. Results of any other analyses performed, including subgroup
    analyses and adjusted analyses, distinguishing pre-specified from
    exploratory
  \item[$\square$]
    19. All important harms or unintended effects in each group
  \end{itemize}
\item[$\square$]
  \textbf{Discussion}

  \begin{itemize}
  \tightlist
  \item[$\square$]
    20. Trial limitations, addressing sources of potential bias,
    imprecision, and, if relevant, multiplicity of analyses
  \item[$\square$]
    21. Generalisability (external validity, applicability) of the trial
    findings
  \item[$\square$]
    22. Interpretation consistent with results, balancing benefits and
    harms, and considering other relevant evidence
  \end{itemize}
\item[$\square$]
  \textbf{Other information}

  \begin{itemize}
  \tightlist
  \item[$\square$]
    23. Registration number and name of trial registry
  \item[$\square$]
    24. Where the full trial protocol can be accessed, if available
  \item[$\square$]
    25. Sources of funding and other support (e.g., supply of drugs),
    role of funders
  \end{itemize}
\end{itemize}

\subsubsection{Notes}\label{notes}

{Reviewer notes}

\end{Form}

\end{document}
