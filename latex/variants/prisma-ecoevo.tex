% Pandoc LaTeX template for checklists with basic PDF forms
\documentclass[11pt]{article}
\usepackage[margin=1in]{geometry}
\usepackage{hyperref}
\usepackage{amssymb}
% Optional AcroTeX eForms for richer widgets
\usepackage{eforms}
\hypersetup{colorlinks=true,linkcolor=blue,urlcolor=blue}
\usepackage{enumitem}
\setlist[itemize]{left=1.2em}
\setlist{noitemsep}
\def\tightlist{}

% Abstraction layer for form widgets
  % eForms uses \checkBox and \textField
  \newcommand{\mkCheckBox}[2][]{\checkBox[#1]{#2}}
  \newcommand{\mkTextField}[2][]{\textField[#1]{#2}}

% Enable form environment for interactive fields
% Usage in content requires raw LaTeX or a pandoc Lua filter to insert \CheckBox/\TextField
\begin{document}

\begin{center}
{\LARGE PRISMA-EcoEvo: The PRISMA Extension for Systematic Reviews in
Ecology and Evolution}\\[4pt]
\normalsize Date: 2021-06-29
\end{center}
\vspace{1em}

% Begin PDF form region
\begin{Form}

\section{PRISMA-EcoEvo Checklist}\label{prisma-ecoevo-checklist}

This checklist is an extension of the PRISMA 2020 statement for
systematic reviews in ecology and evolution.

\subsection{Title and Abstract}\label{title-and-abstract}

\begin{itemize}
\tightlist
\item[$\square$]
  \textbf{1.1} Identify the review as a systematic review,
  meta-analysis, or both.
\item[$\square$]
  \textbf{1.2} Summarise the aims and scope of the review.
\item[$\square$]
  \textbf{1.3} Describe the data set.
\item[$\square$]
  \textbf{1.4} State the results of the primary outcome.
\item[$\square$]
  \textbf{1.5} State conclusions.
\item[$\square$]
  \textbf{1.6} State limitations.
\end{itemize}

\subsection{Aims and Questions}\label{aims-and-questions}

\begin{itemize}
\tightlist
\item[$\square$]
  \textbf{2.1} Provide a rationale for the review.
\item[$\square$]
  \textbf{2.2} Reference any previous reviews or meta-analyses on the
  topic.
\item[$\square$]
  \textbf{2.3} State the aims and scope of the review (including its
  generality).
\item[$\square$]
  \textbf{2.4} State the primary questions the review addresses
  (e.g.~which moderators were tested).
\item[$\square$]
  \textbf{2.5} Describe whether effect sizes were derived from
  experimental and/or observational comparisons.
\end{itemize}

\subsection{Review Registration}\label{review-registration}

\begin{itemize}
\tightlist
\item[$\square$]
  \textbf{3.1} Register review aims, hypotheses (if applicable), and
  methods in a time-stamped and publicly accessible archive and provide
  a link to the registration in the methods section of the manuscript.
  Ideally registration occurs before the search, but it can be done at
  any stage before data analysis.
\item[$\square$]
  \textbf{3.2} Describe deviations from the registered aims and methods.
\item[$\square$]
  \textbf{3.3} Justify deviations from the registered aims and methods.
\end{itemize}

\subsection{Eligibility Criteria}\label{eligibility-criteria}

\begin{itemize}
\tightlist
\item[$\square$]
  \textbf{4.1} Report the specific criteria used for including or
  excluding studies when screening titles and/or abstracts, and full
  texts, according to the aims of the systematic review (e.g.~study
  design, taxa, data availability).
\item[$\square$]
  \textbf{4.2} Justify criteria, if necessary (i.e.~not obvious from
  aims and scope).
\end{itemize}

\subsection{Finding Studies}\label{finding-studies}

\begin{itemize}
\tightlist
\item[$\square$]
  \textbf{5.1} Define the type of search (e.g.~comprehensive search,
  representative sample).
\item[$\square$]
  \textbf{5.2} State what sources of information were sought
  (e.g.~published and unpublished studies, personal communications).
\item[$\square$]
  \textbf{5.3} Include, for each database searched, the exact search
  strings used, with keyword combinations and Boolean operators.
\item[$\square$]
  \textbf{5.4} Provide enough information to repeat the equivalent
  search (if possible), including the timespan covered (start and end
  dates).
\end{itemize}

\subsection{Study Selection}\label{study-selection}

\begin{itemize}
\tightlist
\item[$\square$]
  \textbf{6.1} Describe how studies were selected for inclusion at each
  stage of the screening process (e.g.~use of decision trees, screening
  software).
\item[$\square$]
  \textbf{6.2} Report the number of people involved and how they
  contributed (e.g.~independent parallel screening).
\end{itemize}

\subsection{Data Collection Process}\label{data-collection-process}

\begin{itemize}
\tightlist
\item[$\square$]
  \textbf{7.1} Describe where in the reports data were collected from
  (e.g.~text or figures).
\item[$\square$]
  \textbf{7.2} Describe how data were collected (e.g.~software used to
  digitize figures, external data sources).
\item[$\square$]
  \textbf{7.3} Describe moderator variables that were constructed from
  collected data (e.g.~number of generations calculated from years and
  average generation time).
\item[$\square$]
  \textbf{7.4} Report how missing or ambiguous information was dealt
  with during data collection (e.g.~authors of original studies were
  contacted for missing descriptive statistics, and/or effect sizes were
  calculated from test statistics).
\item[$\square$]
  \textbf{7.5} Report who collected data.
\item[$\square$]
  \textbf{7.6} State the number of extractions that were checked for
  accuracy by co-authors.
\end{itemize}

\subsection{Data Items}\label{data-items}

\begin{itemize}
\tightlist
\item[$\square$]
  \textbf{8.1} Describe the key data sought from each study.
\item[$\square$]
  \textbf{8.2} Describe items that do not appear in the main results, or
  which could not be extracted due to insufficient information.
\item[$\square$]
  \textbf{8.3} Describe main assumptions or simplifications that were
  made (e.g.~categorising both `length' and `mass' as `morphology').
\item[$\square$]
  \textbf{8.4} Describe the type of replication unit (e.g.~individuals,
  broods, study sites).
\end{itemize}

\subsection{Assessment of Individual Study
Quality}\label{assessment-of-individual-study-quality}

\begin{itemize}
\tightlist
\item[$\square$]
  \textbf{9.1} Describe whether the quality of studies included in the
  systematic review or meta-analysis was assessed (e.g.~blinded data
  collection, reporting quality, experimental versus observational).
\item[$\square$]
  \textbf{9.2} Describe how information about study quality was
  incorporated into analyses (e.g.~meta-regression and/or sensitivity
  analysis).
\end{itemize}

\subsection{Effect Size Measures}\label{effect-size-measures}

\begin{itemize}
\tightlist
\item[$\square$]
  \textbf{10.1} Describe effect size(s) used.
\item[$\square$]
  \textbf{10.2} Provide a reference to the equation of each calculated
  effect size (e.g.~standardised mean difference, log response ratio)
  and (if applicable) its sampling variance.
\item[$\square$]
  \textbf{10.3} If no reference exists, derive the equations for each
  effect size and state the assumed sampling distribution(s).
\end{itemize}

\subsection{Missing Data}\label{missing-data}

\begin{itemize}
\tightlist
\item[$\square$]
  \textbf{11.1} Describe any steps taken to deal with missing data
  during analysis (e.g.~imputation, complete case, subset analysis).
\item[$\square$]
  \textbf{11.2} Justify the decisions made to deal with missing data.
\end{itemize}

\subsection{Meta-analytic Model
Description}\label{meta-analytic-model-description}

\begin{itemize}
\tightlist
\item[$\square$]
  \textbf{12.1} Describe the models used for synthesis of effect sizes.
\item[$\square$]
  \textbf{12.2} The most common approach in ecology and evolution will
  be a random-effects model, often with a hierarchical/multilevel
  structure. If other types of models are chosen (e.g.~common/fixed
  effects model, unweighted model), provide justification for this
  choice.
\end{itemize}

\subsection{Software}\label{software}

\begin{itemize}
\tightlist
\item[$\square$]
  \textbf{13.1} Describe the statistical platform used for inference
  (e.g.~R).
\item[$\square$]
  \textbf{13.2} Describe the packages used to run models.
\item[$\square$]
  \textbf{13.3} Describe the functions used to run models.
\item[$\square$]
  \textbf{14.4} Describe any arguments that differed from the default
  settings.
\item[$\square$]
  \textbf{13.5} Describe the version numbers of all software used.
\end{itemize}

\subsection{Non-independence}\label{non-independence}

\begin{itemize}
\tightlist
\item[$\square$]
  \textbf{14.1} Describe the types of non-independence encountered
  (e.g.~phylogenetic, spatial, multiple measurements over time).
\item[$\square$]
  \textbf{14.2} Describe how non-independence has been handled.
\item[$\square$]
  \textbf{14.3} Justify decisions made.
\end{itemize}

\subsection{Meta-regression and Model
Selection}\label{meta-regression-and-model-selection}

\begin{itemize}
\tightlist
\item[$\square$]
  \textbf{15.1} Provide a rationale for the inclusion of moderators
  (covariates) that were evaluated in meta-regression models.
\item[$\square$]
  \textbf{15.2} Justify the number of parameters estimated in models, in
  relation to the number of effect sizes and studies (e.g.~interaction
  terms were not included due to insufficient sample sizes).
\item[$\square$]
  \textbf{15.3} Describe any process of model selection.
\end{itemize}

\subsection{Publication Bias and Sensitivity
Analysis}\label{publication-bias-and-sensitivity-analysis}

\begin{itemize}
\tightlist
\item[$\square$]
  \textbf{16.1} Describe assessments of the risk of bias due to missing
  results (e.g.~publication, time-lag, and taxonomic biases).
\item[$\square$]
  \textbf{16.2} Describe any steps taken to investigate the effects of
  such biases (if present).
\item[$\square$]
  \textbf{16.3} Describe any other analyses of robustness of the
  results, e.g.~due to effect size choice, weighting or analytical model
  assumptions, inclusion or exclusion of subsets of the data, or the
  inclusion of alternative moderator variables in meta-regressions.
\end{itemize}

\subsection{Clarification of Post Hoc
Analyses}\label{clarification-of-post-hoc-analyses}

\begin{itemize}
\tightlist
\item[$\square$]
  \textbf{17.1} When hypotheses were formulated after data analysis,
  this should be acknowledged.
\end{itemize}

\subsection{Metadata, Data, and Code}\label{metadata-data-and-code}

\begin{itemize}
\tightlist
\item[$\square$]
  \textbf{18.1} Share metadata (i.e.~data descriptions).
\item[$\square$]
  \textbf{18.2} Share data required to reproduce the results presented
  in the manuscript.
\item[$\square$]
  \textbf{18.3} Share additional data, including information that was
  not presented in the manuscript (e.g.~raw data used to calculate
  effect sizes, descriptions of where data were located in papers).
\item[$\square$]
  \textbf{18.4} Share analysis scripts (or, if a software package with
  graphical user interface (GUI) was used, then describe full model
  specification and fully specify choices).
\end{itemize}

\subsection{Results of Study Selection
Process}\label{results-of-study-selection-process}

\begin{itemize}
\tightlist
\item[$\square$]
  \textbf{19.1} Report the number of studies screened.
\item[$\square$]
  \textbf{19.2} Report the number of studies excluded at each stage of
  screening.
\item[$\square$]
  \textbf{19.3} Report brief reasons for exclusion from the full text
  stage.
\item[$\square$]
  \textbf{19.4} Present a Preferred Reporting Items for Systematic
  Reviews and Meta-Analyses (PRISMA)-like flowchart
  (www.prisma-statement.org).
\end{itemize}

\subsection{Sample Sizes and Study
Characteristics}\label{sample-sizes-and-study-characteristics}

\begin{itemize}
\tightlist
\item[$\square$]
  \textbf{20.1} Report the number of studies and effect sizes for data
  included in meta-analyses.
\item[$\square$]
  \textbf{20.2} Report the number of studies and effect sizes for
  subsets of data included in meta-regressions.
\item[$\square$]
  \textbf{20.3} Provide a summary of key characteristics for reported
  outcomes (either in text or figures; e.g.~one quarter of effect sizes
  reported for vertebrates and the rest invertebrates).
\item[$\square$]
  \textbf{20.4} Provide a summary of limitations of included moderators
  (e.g.~collinearity and overlap between moderators).
\item[$\square$]
  \textbf{20.5} Provide a summary of characteristics related to
  individual study quality (risk of bias).
\end{itemize}

\subsection{Meta-analysis}\label{meta-analysis}

\begin{itemize}
\tightlist
\item[$\square$]
  \textbf{21.1} Provide a quantitative synthesis of results across
  studies, including estimates for the mean effect size, with
  confidence/credible intervals.
\end{itemize}

\subsection{Heterogeneity}\label{heterogeneity}

\begin{itemize}
\tightlist
\item[$\square$]
  \textbf{22.1} Report indicators of heterogeneity in the estimated
  effect (e.g.~I2, tau2 and other variance components).
\end{itemize}

\subsection{Meta-regression}\label{meta-regression}

\begin{itemize}
\tightlist
\item[$\square$]
  \textbf{23.1} Provide estimates of meta-regression slopes
  (i.e.~regression coefficients) and confidence/credible intervals.
\item[$\square$]
  \textbf{23.2} Include estimates and confidence/credible intervals for
  all moderator variables that were assessed (i.e.~complete reporting).
\item[$\square$]
  \textbf{23.3} Report interactions, if they were included.
\item[$\square$]
  \textbf{23.4} Describe outcomes from model selection, if done (e.g.~R2
  and AIC).
\end{itemize}

\subsection{Outcomes of Publication Bias and Sensitivity
Analysis}\label{outcomes-of-publication-bias-and-sensitivity-analysis}

\begin{itemize}
\tightlist
\item[$\square$]
  \textbf{24.1} Provide results for the assessments of the risks of bias
  (e.g.~Egger's regression, funnel plots).
\item[$\square$]
  \textbf{24.2} Provide results for the robustness of the review's
  results (e.g.~subgroup analyses, meta-regression of study quality,
  results from alternative methods of analysis, and temporal trends).
\end{itemize}

\subsection{Discussion}\label{discussion}

\begin{itemize}
\tightlist
\item[$\square$]
  \textbf{25.1} Summarise the main findings in terms of the magnitude of
  effect.
\item[$\square$]
  \textbf{25.2} Summarise the main findings in terms of the precision of
  effects (e.g.~size of confidence intervals, statistical significance).
\item[$\square$]
  \textbf{25.3} Summarise the main findings in terms of their
  heterogeneity.
\item[$\square$]
  \textbf{25.4} Summarise the main findings in terms of their
  biological/practical relevance.
\item[$\square$]
  \textbf{25.5} Compare results with previous reviews on the topic, if
  available.
\item[$\square$]
  \textbf{25.6} Consider limitations and their influence on the
  generality of conclusions, such as gaps in the available evidence
  (e.g.~taxonomic and geographical research biases).
\end{itemize}

\subsection{Contributions and Funding}\label{contributions-and-funding}

\begin{itemize}
\tightlist
\item[$\square$]
  \textbf{26.1} Provide names, affiliations, and funding sources of all
  co-authors.
\item[$\square$]
  \textbf{26.2} List the contributions of each co-author.
\item[$\square$]
  \textbf{26.3} Provide contact details for the corresponding author.
\item[$\square$]
  \textbf{26.4} Disclose any conflicts of interest.
\end{itemize}

\subsection{References}\label{references}

\begin{itemize}
\tightlist
\item[$\square$]
  \textbf{27.1} Provide a reference list of all studies included in the
  systematic review or meta-analysis.
\item[$\square$]
  \textbf{27.2} List included studies as referenced sources (e.g.~rather
  than listing them in a table or supplement).
\end{itemize}

\end{Form}

\end{document}
