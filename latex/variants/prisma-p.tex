% Pandoc LaTeX template for checklists with basic PDF forms
\documentclass[11pt]{article}
\usepackage[margin=1in]{geometry}
\usepackage{hyperref}
\usepackage{amssymb}
% Optional AcroTeX eForms for richer widgets
\usepackage{eforms}
\hypersetup{colorlinks=true,linkcolor=blue,urlcolor=blue}
\usepackage{enumitem}
\setlist[itemize]{left=1.2em}
\setlist{noitemsep}
\def\tightlist{}

% Abstraction layer for form widgets
  % eForms uses \checkBox and \textField
  \newcommand{\mkCheckBox}[2][]{\checkBox[#1]{#2}}
  \newcommand{\mkTextField}[2][]{\textField[#1]{#2}}

% Enable form environment for interactive fields
% Usage in content requires raw LaTeX or a pandoc Lua filter to insert \CheckBox/\TextField
\begin{document}

\begin{center}
{\LARGE PRISMA-P Checklist}\\[4pt]
\normalsize Date: 2015-01-09
\end{center}
\vspace{1em}

% Begin PDF form region
\begin{Form}

\section{PRISMA-P Checklist}\label{prisma-p-checklist}

\begin{quote}
Scope: Preferred Reporting Items for Systematic review and Meta-Analysis
Protocols.

Reference: See \texttt{source/variants/prisma-p.yml} for canonical link
and provenance.
\end{quote}

\subsection{Instructions}\label{instructions}

\begin{itemize}
\tightlist
\item
  Use the boxes to confirm each reporting item.
\item
  Add reviewer notes under each section as needed.
\end{itemize}

\subsection{Title}\label{title}

\begin{itemize}
\tightlist
\item[$\square$]
  \textbf{1a. Title:} Identification of the report as a protocol of a
  systematic review.
\item[$\square$]
  \textbf{1b. Update:} If the protocol is for an update of an existing
  review, identify as such.
\end{itemize}

\subsection{Abstract}\label{abstract}

\begin{itemize}
\tightlist
\item[$\square$]
  \textbf{2. Abstract:} Provide a structured summary of the protocol,
  including:

  \begin{itemize}
  \tightlist
  \item
    \textbf{Background:} Rationale for the review.
  \item
    \textbf{Methods:} Key elements of the methods, including eligibility
    criteria, information sources, risk of bias assessment, and data
    synthesis.
  \item
    \textbf{Registration:} If registered, provide the registration
    number and registry name.
  \end{itemize}
\end{itemize}

\subsection{Introduction}\label{introduction}

\begin{itemize}
\tightlist
\item[$\square$]
  \textbf{3. Rationale:} Describe the rationale for the review in the
  context of what is already known.
\item[$\square$]
  \textbf{4. Objectives:} Provide an explicit statement of the
  question(s) the review will address with reference to participants,
  interventions, comparators, and outcomes (PICO).
\end{itemize}

\subsection{Methods}\label{methods}

\begin{itemize}
\tightlist
\item[$\square$]
  \textbf{5. Eligibility criteria:} Specify the study characteristics
  (e.g., PICO, study design, setting, time frame) and report
  characteristics (e.g., years considered, language, publication status)
  to be used as criteria for eligibility for the review.
\item[$\square$]
  \textbf{6. Information sources:} Describe all intended information
  sources (e.g., electronic databases, contact with study authors, trial
  registers, or other grey literature sources) with planned dates of
  coverage.
\item[$\square$]
  \textbf{7. Search strategy:} Present a draft of the search strategy to
  be used for at least one electronic database, including planned
  limits, such that it could be repeated.
\item[$\square$]
  \textbf{8. Study records: data management:} Describe the mechanism(s)
  that will be used to manage records and data throughout the review.
\item[$\square$]
  \textbf{9. Study records: selection process:} State the process that
  will be used for selecting studies (e.g., two independent reviewers)
  for inclusion in the review.
\item[$\square$]
  \textbf{10. Study records: data collection process:} Describe the
  process of data extraction, including how it will be done and who will
  do it.
\item[$\square$]
  \textbf{11a. Data items:} List and define all variables for which data
  will be sought, including PICO and other relevant data for the review.
\item[$\square$]
  \textbf{11b. Outcomes and prioritization:} List and define all
  outcomes for which data will be sought. If more than one, prioritize
  and explain the choice of the main outcomes.
\item[$\square$]
  \textbf{12. Risk of bias in individual studies:} Describe the planned
  method for assessing risk of bias in individual studies, including how
  it will be used in data synthesis.
\item[$\square$]
  \textbf{13. Data synthesis:} Describe the planned methods of data
  synthesis, including a description of the summary measures and any
  planned investigation of heterogeneity.
\item[$\square$]
  \textbf{14. Meta-bias(es):} Describe any planned assessment of
  meta-bias(es) (e.g., publication bias, selective reporting within
  studies).
\item[$\square$]
  \textbf{15. Confidence in cumulative evidence:} Describe how the
  strength of the body of evidence will be assessed.
\end{itemize}

\subsection{Other}\label{other}

\begin{itemize}
\tightlist
\item[$\square$]
  \textbf{16. Amendments:} Describe any planned amendments to the
  protocol.
\item[$\square$]
  \textbf{17. Dissemination:} Describe the planned dissemination
  strategy.
\end{itemize}

\subsubsection{Notes}\label{notes}

{Reviewer notes}

\subsection{Provenance}\label{provenance}

\begin{itemize}
\tightlist
\item
  Source: See sidecar metadata in \texttt{source/variants/prisma-p.yml}
\item
  Version: 2015
\item
  License: CC-BY-4.0
\end{itemize}

\end{Form}

\end{document}
