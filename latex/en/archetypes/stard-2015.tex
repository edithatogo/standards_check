% Pandoc LaTeX template for checklists with basic PDF forms
\documentclass[11pt]{article}
\usepackage[margin=1in]{geometry}
\usepackage{hyperref}
\usepackage{amssymb}
% Optional AcroTeX eForms for richer widgets
\hypersetup{colorlinks=true,linkcolor=blue,urlcolor=blue}
\usepackage{enumitem}
\setlist[itemize]{left=1.2em}
\setlist{noitemsep}
\def\tightlist{}

% Abstraction layer for form widgets
  % hyperref provides \CheckBox and \TextField
  \newcommand{\mkCheckBox}[2][]{\CheckBox[#1]{#2}}
  \newcommand{\mkTextField}[2][]{\TextField[#1]{#2}}

% Enable form environment for interactive fields
% Usage in content requires raw LaTeX or a pandoc Lua filter to insert \CheckBox/\TextField
\begin{document}

\begin{center}
{\LARGE }\\[4pt]
\normalsize 
\end{center}
\vspace{1em}

% Begin PDF form region
\begin{Form}

\section{STARD 2015 Checklist}\label{stard-2015-checklist}

\begin{quote}
Scope: Standards for Reporting of Diagnostic Accuracy Studies.

Reference: See \texttt{source/archetypes/stard-2015.yml} for canonical
link and provenance.
\end{quote}

\subsection{Instructions}\label{instructions}

\begin{itemize}
\tightlist
\item
  Use the boxes to confirm each reporting item.
\item
  Add reviewer notes under each section as needed.
\end{itemize}

\subsection{Title or Abstract}\label{title-or-abstract}

\begin{itemize}
\tightlist
\item[$\square$]
  \textbf{1. Identification:} Identification as a study of diagnostic
  accuracy using at least one measure of accuracy (such as sensitivity,
  specificity, predictive values, or AUC).
\end{itemize}

\subsection{Abstract}\label{abstract}

\begin{itemize}
\tightlist
\item[$\square$]
  \textbf{2. Structured summary:} Structured summary of study design,
  methods, results, and conclusions (for specific guidance, see STARD
  for Abstracts).
\end{itemize}

\subsection{Introduction}\label{introduction}

\begin{itemize}
\tightlist
\item[$\square$]
  \textbf{3. Scientific and clinical background:} Scientific and
  clinical background, including the intended use and clinical role of
  the index test.
\item[$\square$]
  \textbf{4. Study objectives and hypotheses:} Study objectives and
  hypotheses.
\end{itemize}

\subsection{Methods}\label{methods}

\begin{itemize}
\tightlist
\item[$\square$]
  \textbf{5. Study design:} Whether data collection was planned before
  the index test and reference standard were performed (prospective
  study) or after (retrospective study).
\item[$\square$]
  \textbf{6. Participants:} Eligibility criteria.
\item[$\square$]
  \textbf{7. Participants:} On what basis potentially eligible
  participants were identified (such as symptoms, results from previous
  tests, inclusion in registry).
\item[$\square$]
  \textbf{8. Participants:} Where and when potentially eligible
  participants were identified (setting, location and dates).
\item[$\square$]
  \textbf{9. Participants:} Whether participants formed a consecutive,
  random or convenience series.
\item[$\square$]
  \textbf{10a. Test methods:} Index test, in sufficient detail to allow
  replication.
\item[$\square$]
  \textbf{10b. Test methods:} Reference standard, in sufficient detail
  to allow replication.
\item[$\square$]
  \textbf{11. Test methods:} Rationale for choosing the reference
  standard (if alternatives exist).
\item[$\square$]
  \textbf{12a. Test methods:} Definition of and rationale for test
  positivity cut-offs or result categories of the index test,
  distinguishing prespecified from exploratory.
\item[$\square$]
  \textbf{12b. Test methods:} Definition of and rationale for test
  positivity cut-offs or result categories of the reference standard,
  distinguishing prespecified from exploratory.
\item[$\square$]
  \textbf{13a. Test methods:} Whether clinical information and reference
  standard results were available to the performers/readers of the index
  test.
\item[$\square$]
  \textbf{13b. Test methods:} Whether clinical information and index
  test results were available to the assessors of the reference
  standard.
\item[$\square$]
  \textbf{14. Analysis:} Methods for estimating or comparing measures of
  diagnostic accuracy.
\item[$\square$]
  \textbf{15. Analysis:} How indeterminate index test or reference
  standard results were handled.
\item[$\square$]
  \textbf{16. Analysis:} How missing data on the index test and
  reference standard were handled.
\item[$\square$]
  \textbf{17. Analysis:} Any analyses of variability in diagnostic
  accuracy, distinguishing prespecified from exploratory.
\item[$\square$]
  \textbf{18. Analysis:} Sample size calculation.
\end{itemize}

\subsection{Results}\label{results}

\begin{itemize}
\tightlist
\item[$\square$]
  \textbf{19. Participants:} Flow of participants, using a diagram.
\item[$\square$]
  \textbf{20. Participants:} Baseline demographic and clinical
  characteristics of participants.
\item[$\square$]
  \textbf{21. Participants:} Distribution of severity of disease in
  those with the target condition; other diagnoses in participants
  without the target condition.
\item[$\square$]
  \textbf{22. Test results:} Time interval from index test to reference
  standard, and any treatment administered between them.
\item[$\square$]
  \textbf{23. Test results:} Cross tabulation of the index test results
  (or their distribution) by the results of the reference standard; for
  continuous results, the distribution of the test results by the
  results of the reference standard.
\item[$\square$]
  \textbf{24. Test results:} Any adverse events from performing the
  index test or the reference standard.
\item[$\square$]
  \textbf{25. Estimates:} Estimates of diagnostic accuracy and their
  precision (such as 95\% confidence intervals).
\item[$\square$]
  \textbf{26. Estimates:} Any analyses of variability, including
  subgroup analyses.
\item[$\square$]
  \textbf{27. Estimates:} The number of indeterminate test results or
  missing data and where they occurred.
\end{itemize}

\subsection{Discussion}\label{discussion}

\begin{itemize}
\tightlist
\item[$\square$]
  \textbf{28. Study limitations:} Study limitations, including sources
  of potential bias, statistical uncertainty, and generalisability.
\item[$\square$]
  \textbf{29. Implications for practice:} Implications for practice,
  including the intended use and clinical role of the index test.
\end{itemize}

\subsection{Other Information}\label{other-information}

\begin{itemize}
\tightlist
\item[$\square$]
  \textbf{30. Registration number and name of registry:} Registration
  number and name of registry.
\item[$\square$]
  \textbf{31. Where the full study protocol can be accessed:} Where the
  full study protocol can be accessed.
\item[$\square$]
  \textbf{32. Funding:} Sources of funding and other support; role of
  funders.
\end{itemize}

{Notes}

\subsection{Provenance}\label{provenance}

\begin{itemize}
\tightlist
\item
  Source: See sidecar metadata in
  \texttt{source/archetypes/stard-2015.yml}
\item
  Version: 2015
\item
  License: CC-BY-4.0
\end{itemize}

\end{Form}

\end{document}
