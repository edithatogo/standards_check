% Pandoc LaTeX template for checklists with basic PDF forms
\documentclass[11pt]{article}
\usepackage[margin=1in]{geometry}
\usepackage{hyperref}
\usepackage{amssymb}
% Optional AcroTeX eForms for richer widgets
\hypersetup{colorlinks=true,linkcolor=blue,urlcolor=blue}
\usepackage{enumitem}
\setlist[itemize]{left=1.2em}
\setlist{noitemsep}
\def\tightlist{}

% Abstraction layer for form widgets
  % hyperref provides \CheckBox and \TextField
  \newcommand{\mkCheckBox}[2][]{\CheckBox[#1]{#2}}
  \newcommand{\mkTextField}[2][]{\TextField[#1]{#2}}

% Enable form environment for interactive fields
% Usage in content requires raw LaTeX or a pandoc Lua filter to insert \CheckBox/\TextField
\begin{document}

\begin{center}
{\LARGE }\\[4pt]
\normalsize 
\end{center}
\vspace{1em}

% Begin PDF form region
\begin{Form}

\section{CHEERS 2022 Checklist}\label{cheers-2022-checklist}

\begin{quote}
Scope: The Consolidated Health Economic Evaluation Reporting Standards
(CHEERS) 2022 statement provides a 28-item checklist to promote
transparent and comprehensive reporting of economic evaluations of
health interventions.

Reference: See \texttt{source/archetypes/cheers-2022.yml} for canonical
link and provenance.
\end{quote}

\subsection{Instructions}\label{instructions}

\begin{itemize}
\tightlist
\item
  Use task list items for checklist boxes; these become interactive
  checkboxes in PDF.
\item
  Use a span with class \texttt{.textfield} for free‑text fields.
\end{itemize}

\subsection{Title and Abstract}\label{title-and-abstract}

\begin{itemize}
\tightlist
\item[$\square$]
  \textbf{Title:} Identify the study as an economic evaluation and
  specify the interventions being compared.
\item[$\square$]
  \textbf{Abstract:} Provide a structured summary detailing the context,
  key methods, results, and alternative analyses.
\end{itemize}

\subsection{Introduction}\label{introduction}

\begin{itemize}
\tightlist
\item[$\square$]
  \textbf{Background and Objectives:} Describe the study's context, the
  research question, and its relevance for policy or practice.
\end{itemize}

\subsection{Methods}\label{methods}

\begin{itemize}
\tightlist
\item[$\square$]
  \textbf{Health Economic Analysis Plan:} Indicate if a health economic
  analysis plan was developed and where it can be accessed.
\item[$\square$]
  \textbf{Study Population:} Describe the characteristics of the study
  population, including demographics and clinical characteristics.
\item[$\square$]
  \textbf{Setting and Location:} Provide relevant contextual information
  that could influence the findings.
\item[$\square$]
  \textbf{Comparators:} Describe the interventions or strategies being
  compared and the rationale for their choice.
\item[$\square$]
  \textbf{Perspective:} State the perspective(s) of the study and the
  reasoning for their selection.
\item[$\square$]
  \textbf{Time Horizon:} Report the time horizon for the study and
  justify its appropriateness.
\item[$\square$]
  \textbf{Discount Rate:} Report the discount rate(s) used and the
  reason for their choice.
\item[$\square$]
  \textbf{Selection of Outcomes:} Describe the outcomes used to measure
  benefits and harms.
\item[$\square$]
  \textbf{Measurement of Outcomes:} Detail how the selected outcomes
  were measured.
\item[$\square$]
  \textbf{Valuation of Outcomes:} Describe the methods and population
  used to measure and value outcomes.
\item[$\square$]
  \textbf{Measurement and Valuation of Resources and Costs:} Explain how
  costs were valued.
\item[$\square$]
  \textbf{Currency, Price Date, and Conversion:} Report the dates of
  estimated resource quantities and unit costs, as well as the currency
  and conversion year.
\item[$\square$]
  \textbf{Rationale and Description of Model:} If a model was used,
  describe it in detail, explain its use, and state if it is publicly
  available.
\item[$\square$]
  \textbf{Analytics and Assumptions:} Describe methods for data
  analysis, statistical transformation, extrapolation, and model
  validation.
\end{itemize}

\subsection{Results}\label{results}

\begin{itemize}
\tightlist
\item[$\square$]
  \textbf{Study Parameters:} Report all parameters used in the analysis,
  including their sources and any assumptions.
\item[$\square$]
  \textbf{Summary of Main Results:} Present the main results, including
  costs, effects, and the incremental cost-effectiveness ratio.
\item[$\square$]
  \textbf{Characterizing Heterogeneity:} Describe any methods used to
  explore how results may vary for different subgroups.
\item[$\square$]
  \textbf{Characterizing Distributional Effects:} Explain how impacts
  are distributed across different individuals or any adjustments made
  for priority populations.
\item[$\square$]
  \textbf{Characterizing Uncertainty:} Describe the methods used to
  characterize any sources of uncertainty in the analysis.
\end{itemize}

\subsection{Discussion}\label{discussion}

\begin{itemize}
\tightlist
\item[$\square$]
  \textbf{Summary of Findings, Limitations, Generalizability, and
  Current Knowledge:} Summarize the findings, discuss limitations, and
  consider the generalizability of the results in the context of current
  knowledge.
\item[$\square$]
  \textbf{Conclusions:} Provide conclusions about the study's findings
  and their implications for policy or practice.
\end{itemize}

\subsection{Other}\label{other}

\begin{itemize}
\tightlist
\item[$\square$]
  \textbf{Conflicts of Interest:} Disclose any potential conflicts of
  interest.
\item[$\square$]
  \textbf{Funding:} Identify the sources of funding for the study.
\item[$\square$]
  \textbf{Approach to Engagement:} Describe the approach to engaging
  with patients and others affected by the study.
\end{itemize}

{Notes}

\subsection{Provenance}\label{provenance}

\begin{itemize}
\tightlist
\item
  Source: See sidecar metadata in
  \texttt{source/archetypes/cheers-2022.yml}
\item
  Version: 2022
\item
  License: CC-BY-4.0
\end{itemize}

\end{Form}

\end{document}
