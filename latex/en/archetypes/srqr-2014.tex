% Pandoc LaTeX template for checklists with basic PDF forms
\documentclass[11pt]{article}
\usepackage[margin=1in]{geometry}
\usepackage{hyperref}
\usepackage{amssymb}
% Optional AcroTeX eForms for richer widgets
\hypersetup{colorlinks=true,linkcolor=blue,urlcolor=blue}
\usepackage{enumitem}
\setlist[itemize]{left=1.2em}
\setlist{noitemsep}
\def\tightlist{}

% Abstraction layer for form widgets
  % hyperref provides \CheckBox and \TextField
  \newcommand{\mkCheckBox}[2][]{\CheckBox[#1]{#2}}
  \newcommand{\mkTextField}[2][]{\TextField[#1]{#2}}

% Enable form environment for interactive fields
% Usage in content requires raw LaTeX or a pandoc Lua filter to insert \CheckBox/\TextField
\begin{document}

\begin{center}
{\LARGE }\\[4pt]
\normalsize 
\end{center}
\vspace{1em}

% Begin PDF form region
\begin{Form}

\section{SRQR Checklist}\label{srqr-checklist}

\begin{quote}
Scope: Standards for Reporting Qualitative Research.

Reference: See \texttt{source/archetypes/srqr-2014.yml} for canonical
link and provenance.
\end{quote}

\subsection{Instructions}\label{instructions}

\begin{itemize}
\tightlist
\item
  Use the boxes to confirm each reporting item.
\item
  Add reviewer notes under each section as needed.
\end{itemize}

\subsection{Title and Abstract}\label{title-and-abstract}

\begin{itemize}
\tightlist
\item[$\square$]
  \textbf{1. Title:} Concise description of the nature and topic of the
  study.
\item[$\square$]
  \textbf{2. Abstract:} Summary of the study, including the problem,
  purpose, methods, findings, and implications.
\end{itemize}

\subsection{Introduction}\label{introduction}

\begin{itemize}
\tightlist
\item[$\square$]
  \textbf{3. Problem formulation:} Description of the problem or
  question and its importance.
\item[$\square$]
  \textbf{4. Purpose or research question:} Purpose of the study or
  research question.
\item[$\square$]
  \textbf{5. Rationale:} Rationale for the study, including the state of
  knowledge on the topic.
\end{itemize}

\subsection{Methods}\label{methods}

\begin{itemize}
\tightlist
\item[$\square$]
  \textbf{6. Qualitative approach and research paradigm:} The
  qualitative approach (e.g., ethnography, grounded theory) and the
  research paradigm (e.g., positivist, constructivist).
\item[$\square$]
  \textbf{7. Researcher characteristics and reflexivity:} The
  researchers' characteristics and their reflexivity on how they might
  have influenced the research.
\item[$\square$]
  \textbf{8. Context:} The context of the study (e.g., setting,
  participants).
\item[$\square$]
  \textbf{9. Sampling strategy:} The sampling strategy used to select
  participants.
\item[$\square$]
  \textbf{10. Ethical considerations:} Ethical considerations, including
  how informed consent was obtained.
\item[$\square$]
  \textbf{11. Data collection methods:} The methods used to collect data
  (e.g., interviews, focus groups, observation).
\item[$\square$]
  \textbf{12. Data analysis:} The methods used to analyze the data.
\item[$\square$]
  \textbf{13. Techniques to enhance trustworthiness:} The techniques
  used to enhance the trustworthiness of the findings.
\end{itemize}

\subsection{Results/Findings}\label{resultsfindings}

\begin{itemize}
\tightlist
\item[$\square$]
  \textbf{14. Main findings:} The main findings of the study.
\item[$\square$]
  \textbf{15. Quotes/excerpts:} The use of quotes or excerpts to support
  the findings.
\item[$\square$]
  \textbf{16. Integration with prior work:} The integration of the
  findings with prior work.
\end{itemize}

\subsection{Discussion}\label{discussion}

\begin{itemize}
\tightlist
\item[$\square$]
  \textbf{17. Discussion of findings:} Discussion of the findings,
  including their implications.
\item[$\square$]
  \textbf{18. Limitations:} The limitations of the study.
\item[$\square$]
  \textbf{19. Conclusions:} The conclusions of the study.
\end{itemize}

\subsection{Other}\label{other}

\begin{itemize}
\tightlist
\item[$\square$]
  \textbf{20. Funding:} The sources of funding for the study.
\item[$\square$]
  \textbf{21. Conflicts of interest:} Any conflicts of interest.
\end{itemize}

\subsubsection{Notes}\label{notes}

{Reviewer notes}

\subsection{Provenance}\label{provenance}

\begin{itemize}
\tightlist
\item
  Source: See sidecar metadata in
  \texttt{source/archetypes/srqr-2014.yml}
\item
  Version: 2014
\item
  License: Copyright, AAMC
\end{itemize}

\end{Form}

\end{document}
