% Pandoc LaTeX template for checklists with basic PDF forms
\documentclass[11pt]{article}
\usepackage[margin=1in]{geometry}
\usepackage{hyperref}
\usepackage{amssymb}
% Optional AcroTeX eForms for richer widgets
\hypersetup{colorlinks=true,linkcolor=blue,urlcolor=blue}
\usepackage{enumitem}
\setlist[itemize]{left=1.2em}
\setlist{noitemsep}
\def\tightlist{}

% Abstraction layer for form widgets
  % hyperref provides \CheckBox and \TextField
  \newcommand{\mkCheckBox}[2][]{\CheckBox[#1]{#2}}
  \newcommand{\mkTextField}[2][]{\TextField[#1]{#2}}

% Enable form environment for interactive fields
% Usage in content requires raw LaTeX or a pandoc Lua filter to insert \CheckBox/\TextField
\begin{document}

\begin{center}
{\LARGE }\\[4pt]
\normalsize 
\end{center}
\vspace{1em}

% Begin PDF form region
\begin{Form}

\section{STROBE Checklist}\label{strobe-checklist}

\begin{quote}
Scope: STrengthening the Reporting of OBservational studies in
Epidemiology.

Reference: See \texttt{source/archetypes/strobe-2007.yml} for canonical
link and provenance.
\end{quote}

\subsection{Instructions}\label{instructions}

\begin{itemize}
\tightlist
\item
  Use the boxes to confirm each reporting item.
\item
  Add reviewer notes under each section as needed.
\end{itemize}

\subsection{Title and abstract}\label{title-and-abstract}

\begin{itemize}
\tightlist
\item[$\square$]
  \textbf{1. Title and abstract:} Indicate the study's design with a
  commonly used term in the title or the abstract.
\end{itemize}

\subsection{Introduction}\label{introduction}

\begin{itemize}
\tightlist
\item[$\square$]
  \textbf{2. Background/rationale:} Explain the scientific background
  and rationale for the investigation being reported.
\item[$\square$]
  \textbf{3. Objectives:} State specific objectives, including any
  prespecified hypotheses.
\end{itemize}

\subsection{Methods}\label{methods}

\begin{itemize}
\tightlist
\item[$\square$]
  \textbf{4. Study design:} Present key elements of study design early
  in the paper.
\item[$\square$]
  \textbf{5. Setting:} Describe the setting, locations, and relevant
  dates, including periods of recruitment, exposure, follow-up, and data
  collection.
\item[$\square$]
  \textbf{6. Participants:} Give the eligibility criteria, and the
  sources and methods of selection of participants.
\item[$\square$]
  \textbf{7. Variables:} Clearly define all outcomes, exposures,
  predictors, potential confounders, and effect modifiers. Give
  diagnostic criteria, if applicable.
\item[$\square$]
  \textbf{8. Data sources/ measurement:} For each variable of interest,
  give sources of data and details of methods of assessment
  (measurement).
\item[$\square$]
  \textbf{9. Bias:} Describe any efforts to address potential sources of
  bias.
\item[$\square$]
  \textbf{10. Study size:} Explain how the study size was arrived at.
\item[$\square$]
  \textbf{11. Quantitative variables:} Explain how quantitative
  variables were handled in the analyses. If applicable, describe which
  groupings were chosen and why.
\item[$\square$]
  \textbf{12. Statistical methods:} Describe all statistical methods,
  including those used to control for confounding.
\end{itemize}

\subsection{Results}\label{results}

\begin{itemize}
\tightlist
\item[$\square$]
  \textbf{13. Participants:} Report numbers of individuals at each stage
  of study---eg, numbers potentially eligible, examined for eligibility,
  confirmed eligible, included in the study, completing follow-up, and
  analysed.
\item[$\square$]
  \textbf{14. Descriptive data:} Give characteristics of study
  participants (eg, demographic, clinical, social) and information on
  exposures and potential confounders.
\item[$\square$]
  \textbf{15. Outcome data:} Report numbers of outcome events or summary
  measures over time.
\item[$\square$]
  \textbf{16. Main results:} Give unadjusted estimates and, if
  applicable, confounder-adjusted estimates and their precision (eg,
  95\% confidence interval). Make clear which confounders were adjusted
  for and why they were included.
\item[$\square$]
  \textbf{17. Other analyses:} Report other analyses done---eg, analyses
  of subgroups and interactions, and sensitivity analyses.
\end{itemize}

\subsection{Discussion}\label{discussion}

\begin{itemize}
\tightlist
\item[$\square$]
  \textbf{18. Key results:} Summarise key results with reference to
  study objectives.
\item[$\square$]
  \textbf{19. Limitations:} Discuss limitations of the study, taking
  into account sources of potential bias or imprecision. Discuss both
  direction and magnitude of any potential bias.
\item[$\square$]
  \textbf{20. Interpretation:} Give a cautious overall interpretation of
  results considering objectives, limitations, multiplicity of analyses,
  results from similar studies, and other relevant evidence.
\item[$\square$]
  \textbf{21. Generalisability:} Discuss the generalisability (external
  validity) of the study results.
\end{itemize}

\subsection{Other information}\label{other-information}

\begin{itemize}
\tightlist
\item[$\square$]
  \textbf{22. Funding:} Give the source of funding and the role of the
  funders for the present study and, if applicable, for the original
  study on which the present article is based.
\end{itemize}

\subsubsection{Notes}\label{notes}

{Reviewer notes}

\subsection{Provenance}\label{provenance}

\begin{itemize}
\tightlist
\item
  Source: See sidecar metadata in
  \texttt{source/archetypes/strobe-2007.yml}
\item
  Version: 2007
\item
  License: CC BY 4.0
\end{itemize}

\end{Form}

\end{document}
