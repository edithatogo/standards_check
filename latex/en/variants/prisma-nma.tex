% Pandoc LaTeX template for checklists with basic PDF forms
\documentclass[11pt]{article}
\usepackage[margin=1in]{geometry}
\usepackage{hyperref}
\usepackage{amssymb}
% Optional AcroTeX eForms for richer widgets
\hypersetup{colorlinks=true,linkcolor=blue,urlcolor=blue}
\usepackage{enumitem}
\setlist[itemize]{left=1.2em}
\setlist{noitemsep}
\def\tightlist{}

% Abstraction layer for form widgets
  % hyperref provides \CheckBox and \TextField
  \newcommand{\mkCheckBox}[2][]{\CheckBox[#1]{#2}}
  \newcommand{\mkTextField}[2][]{\TextField[#1]{#2}}

% Enable form environment for interactive fields
% Usage in content requires raw LaTeX or a pandoc Lua filter to insert \CheckBox/\TextField
\begin{document}

\begin{center}
{\LARGE }\\[4pt]
\normalsize 
\end{center}
\vspace{1em}

% Begin PDF form region
\begin{Form}

\section{PRISMA-NMA Checklist}\label{prisma-nma-checklist}

\subsection{Instructions}\label{instructions}

\begin{itemize}
\tightlist
\item
  Use the boxes to confirm each reporting item.
\item
  Add reviewer notes under each section as needed.
\end{itemize}

\subsection{Title}\label{title}

\begin{itemize}
\tightlist
\item[$\square$]
  \textbf{1. Title:} Identify the report as a systematic review
  incorporating a network meta-analysis (or related form of
  meta-analysis).
\end{itemize}

\subsection{Abstract}\label{abstract}

\begin{itemize}
\tightlist
\item[$\square$]
  \textbf{2. Structured Summary:} Provide a structured summary
  including, as applicable:

  \begin{itemize}
  \tightlist
  \item
    \textbf{Background:} main objectives
  \item
    \textbf{Methods:} data sources; study eligibility criteria,
    participants, and interventions; study appraisal; and synthesis
    methods, such as network meta-analysis.
  \item
    \textbf{Results:} number of studies and participants identified;
    summary estimates with corresponding confidence/credible intervals;
    treatment rankings may also be discussed. Authors may choose to
    summarize pairwise comparisons against a chosen treatment included
    in their analyses for brevity.
  \item
    \textbf{Discussion/Conclusions:} limitations; conclusions and
    implications of findings.
  \item
    \textbf{Other:} primary source of funding; systematic review
    registration number with registry name.
  \end{itemize}
\end{itemize}

\subsection{Introduction}\label{introduction}

\begin{itemize}
\tightlist
\item[$\square$]
  \textbf{3. Rationale:} Describe the rationale for the review in the
  context of what is already known, including mention of why a network
  meta-analysis has been conducted.
\item[$\square$]
  \textbf{4. Objectives:} Provide an explicit statement of questions
  being addressed, with reference to participants, interventions,
  comparisons, outcomes, and study design (PICOS).
\end{itemize}

\subsection{Methods}\label{methods}

\begin{itemize}
\tightlist
\item[$\square$]
  \textbf{5. Protocol and Registration:} Indicate whether a review
  protocol exists and if and where it can be accessed (e.g., Web
  address); and, if available, provide registration information,
  including registration number.
\item[$\square$]
  \textbf{6. Eligibility Criteria:} Specify study characteristics (e.g.,
  PICOS, length of follow-up) and report characteristics (e.g., years
  considered, language, publication status) used as criteria for
  eligibility, giving rationale. Clearly describe eligible treatments
  included in the treatment network, and note whether any have been
  clustered or merged into the same node (with justification).
\item[$\square$]
  \textbf{7. Information Sources:} Describe all information sources
  (e.g., databases with dates of coverage, contact with study authors to
  identify additional studies) in the search and date last searched.
\item[$\square$]
  \textbf{8. Search:} Present full electronic search strategy for at
  least one database, including any limits used, such that it could be
  repeated.
\item[$\square$]
  \textbf{9. Study Selection:} State the process for selecting studies
  (i.e., screening, eligibility, included in systematic review, and, if
  applicable, included in the meta-analysis).
\item[$\square$]
  \textbf{10. Data Collection Process:} Describe method of data
  extraction from reports (e.g., piloted forms, independently, in
  duplicate) and any processes for obtaining and confirming data from
  investigators.
\item[$\square$]
  \textbf{11. Data Items:} List and define all variables for which data
  were sought (e.g., PICOS, funding sources) and any assumptions and
  simplifications made.
\item[$\square$]
  \textbf{S1. Geometry of the Network:} Describe methods used to explore
  the geometry of the treatment network under study and potential biases
  related to it. This should include how the evidence base has been
  graphically summarized for presentation, and what characteristics were
  compiled and used to describe the evidence base to readers.
\item[$\square$]
  \textbf{12. Risk of Bias within Individual Studies:} Describe methods
  used for assessing risk of bias of individual studies (including
  specification of whether this was done at the study or outcome level),
  and how this information is to be used in any data synthesis.
\item[$\square$]
  \textbf{13. Summary Measures:} State the principal summary measures
  (e.g., risk ratio, difference in means). Also describe the use of
  additional summary measures assessed, such as treatment rankings and
  surface under the cumulative ranking curve (SUCRA) values, as well as
  modified approaches used to present summary findings from
  meta-analyses.
\item[$\square$]
  \textbf{14. Planned Methods of Analysis:} Describe the methods of
  handling data and combining results of studies for each network
  meta-analysis. This should include, but not be limited to:

  \begin{itemize}
  \tightlist
  \item
    Handling of multi-arm trials;
  \item
    Selection of variance structure;
  \item
    Selection of prior distributions in Bayesian analyses; and
  \item
    Assessment of model fit.
  \end{itemize}
\item[$\square$]
  \textbf{S2. Assessment of Inconsistency:} Describe the statistical
  methods used to evaluate the agreement of direct and indirect evidence
  in the treatment network(s) studied. Describe efforts taken to address
  its presence when found.
\item[$\square$]
  \textbf{15. Risk of Bias across Studies:} Specify any assessment of
  risk of bias that may affect the cumulative evidence (e.g.,
  publication bias, selective reporting within studies).
\item[$\square$]
  \textbf{16. Additional Analyses:} Describe methods of additional
  analyses if done, indicating which were pre-specified. This may
  include, but not be limited to, the following:

  \begin{itemize}
  \tightlist
  \item
    Sensitivity or subgroup analyses;
  \item
    Meta-regression analyses;
  \item
    Alternative formulations of the treatment network;
  \item
    Use of alternative prior distributions for Bayesian analyses (if
    applicable).
  \end{itemize}
\end{itemize}

\subsection{Results}\label{results}

\begin{itemize}
\tightlist
\item[$\square$]
  \textbf{17. Study Selection:} Give numbers of studies screened,
  assessed for eligibility, and included in the review, with reasons for
  exclusions at each stage, ideally with a flow diagram.
\item[$\square$]
  \textbf{S3. Presentation of Network Structure:} Provide a network
  graph of the included studies to enable visualization of the geometry
  of the treatment network.
\item[$\square$]
  \textbf{S4. Summary of Network Geometry:} Provide a brief overview of
  characteristics of the treatment network. This may include commentary
  on the abundance of trials and randomized patients for the different
  interventions and pairwise comparisons in the network, gaps of
  evidence in the treatment network, and potential biases reflected by
  the network structure.
\item[$\square$]
  \textbf{18. Study Characteristics:} For each study, present
  characteristics for which data were extracted (e.g., study size,
  PICOS, follow-up period) and provide the citations.
\item[$\square$]
  \textbf{19. Risk of Bias within Studies:} Present data on risk of bias
  of each study and, if available, any outcome level assessment.
\item[$\square$]
  \textbf{20. Results of Individual Studies:} For all outcomes
  considered (benefits or harms), present, for each study: 1) simple
  summary data for each intervention group, and 2) effect estimates and
  confidence intervals. Modified approaches may be needed to deal with
  information from larger networks.
\item[$\square$]
  \textbf{21. Synthesis of Results:} Present results of each
  meta-analysis done, including confidence/credible intervals. In larger
  networks, authors may focus on comparisons versus a particular
  comparator (e.g.~placebo or standard care), with full findings
  presented in an appendix. League tables and forest plots may be
  considered to summarize pairwise comparisons. If additional summary
  measures were explored (such as treatment rankings), these should also
  be presented.
\item[$\square$]
  \textbf{S5. Exploration for Inconsistency:} Describe results from
  investigations of inconsistency. This may include such information as
  measures of model fit to compare consistency and inconsistency models,
  P values from statistical tests, or summary of inconsistency estimates
  from different parts of the treatment network.
\item[$\square$]
  \textbf{22. Risk of Bias across Studies:} Present results of any
  assessment of risk of bias across studies for the evidence base being
  studied.
\item[$\square$]
  \textbf{23. Results of Additional Analyses:} Give results of
  additional analyses, if done (e.g., sensitivity or subgroup analyses,
  meta-regression analyses, alternative network geometries studied,
  alternative choice of prior distributions for Bayesian analyses, and
  so forth).
\end{itemize}

\subsection{Discussion}\label{discussion}

\begin{itemize}
\tightlist
\item[$\square$]
  \textbf{24. Summary of Evidence:} Summarize the main findings,
  including the strength of evidence for each main outcome; consider
  their relevance to key groups (e.g., healthcare providers, users, and
  policy-makers).
\item[$\square$]
  \textbf{25. Limitations:} Discuss limitations at study and outcome
  level (e.g., risk of bias), and at review level (e.g., incomplete
  retrieval of identified research, reporting bias). Comment on the
  validity of the assumptions, such as transitivity and consistency.
  Comment on any concerns regarding network geometry (e.g., avoidance of
  certain comparisons).
\item[$\square$]
  \textbf{26. Conclusions:} Provide a general interpretation of the
  results in the context of other evidence, and implications for future
  research.
\end{itemize}

\subsection{Funding}\label{funding}

\begin{itemize}
\tightlist
\item[$\square$]
  \textbf{27. Funding:} Describe sources of funding for the systematic
  review and other support (e.g., supply of data); role of funders for
  the systematic review. This should also include information regarding
  whether funding has been received from manufacturers of treatments in
  the network and/or whether some of the authors are content experts
  with professional conflicts of interest that could affect use of
  treatments in the network.
\end{itemize}

\subsubsection{Notes}\label{notes}

{Reviewer notes}

\subsection{Provenance}\label{provenance}

\begin{itemize}
\tightlist
\item
  Source: See sidecar metadata in
  \texttt{source/variants/prisma-nma.yml}
\item
  Version: 2015
\item
  License: CC-BY-4.0
\end{itemize}

\end{Form}

\end{document}
